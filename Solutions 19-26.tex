\documentclass[11pt]{article}
\usepackage{amsmath,amssymb,latexsym}

\parindent 0in
\addtolength{\textwidth}{52mm}
\addtolength{\oddsidemargin}{-28mm}
\addtolength{\evensidemargin}{-26mm}
\addtolength{\topmargin}{-20mm}
\addtolength{\textheight}{40mm}

\newcommand{\ds}{\displaystyle}
\renewcommand{\vec}[1]{\mathbf{#1}}
\newcommand{\vc}[1]{\begin{pmatrix}#1\end{pmatrix}}
\newcommand{\tmat}[1]{{\left(\begin{array}{rrr}#1\end{array}\right)}}
\newcommand{\qmat}[1]{{\left(\begin{array}{rrrr}#1\end{array}\right)}}
\newcommand{\pmat}[1]{{\left(\begin{array}{rrrrr}#1\end{array}\right)}}
\newcommand{\proof}{{\sc Proof.}\quad}
\newcommand{\qed}{\quad$\square$}

\newcommand{\moveup}{\begin{picture}(0,0)(0,0)\end{picture}\vspace*{-8.15mm}}
\newcommand{\upabit}{\begin{picture}(0,0)(0,0)\end{picture}\vspace*{-5mm}}

\newcommand{\ra}[1]{\xrightarrow{#1}}
\newcommand{\dra}[2]{\begin{array}{c}\xrightarrow{\text{\scriptsize$#1$}}\\[-2mm]{\text{\scriptsize$#2$}}\end{array}}


\date{}
\author{}
\title{\sc Solutions to MATH3411 Problems 19--26}


\begin{document} \maketitle

\vspace*{-10mm}

\noindent{\bf 19.}\upabit
\begin{itemize}
  \item[{a)}] {\bf Proof}.
      Let $\vec{x}\in C$ be a non-zero codeword with minimal weight:
      $w(\vec{x}) = w(C)$.\\
      Since $C$ is linear, it contains the zero vector $\vec{0}$,
      so
      \[
          d(C)
        = \min\{d(\vec{u},\vec{v})\::\: \vec{u},\vec{v}\in C,\,\vec{x}\neq\vec{y}\}
        \leq d(\vec{x},\vec{0})
        = w(\vec{x})
        = w(C)\,.
      \]
      Similarly, let $\vec{u},\vec{v}\in C$ be distinct codewords
      with minimal distance between them: $d(\vec{u},\vec{v}) = d(C)$.\\
      Since $C$ is linear, it contains $\vec{u}-\vec{v}$,
      so
      \[
          w(C)
        = \min\{w(\vec{z})\::\: \vec{z}\in C,\,\vec{z}\neq\vec{0}\}
        \leq w(\vec{u}-\vec{y})
        = d(\vec{u}-\vec{y},\vec{0})
        = d(\vec{u},\vec{y})
        = d(C)\,.
      \]
      We see that $d(C)\leq w(C)$ and that $w(C)\leq d(C)$, so $d(C) = w(C)$. \hfill$\square$
  \item[{b)}] {\bf Proof}.
      From Part a), we know that $d = d(C) = w(C)$,
      so there is a codeword $\vec{v}\in C$ with $w(\vec{v}) = d$.\\
      Let $I$ be the set of $d$ positions of $\vec{v}$'s non-zero entries: $i\in I$ if and only if $v_i\neq 0$.\\
      Since $v\in C$, we see that $H\vec{v} = 0$,
      or in terms of $H$'s columns $h_i$,
      $\sum_{i\in I} v_ih_i = \vec{0}$.\\
      We see that the $d$ columns $\{h_i\::\: i\in I\}$ are linearly independent.\\
      In other words, the minimal number of dependent columns of $H$ is at most $d$.

      Conversely, consider a minimal number of linearly dependent columns of $H$,
      say $\{h_i\::\: i\in I\}$.\\
      Since they are linearly dependent,
      we can find $d$ non-zero values $\{a_i\::\: i\in I\}$
      that $\sum_{i\in I} a_ih_i = \vec{0}$.
      Define $\vec{v}$ to be the vector with entries $h_i = a_i$ if $i\in I$ and $h_i = 0$ otherwise.
      Then $H\vec{v} = \vec{0}$,
      so $\vec{v} \in C$.
      Also, $d \leq w(\vec{v}) = |I|$.
      Since $|I|$ is the minimum numbers of linearly dependent columns of $H$,
      we see that $d$ is the minimum numbers of linearly dependent columns of $H$.
      \hfill$\square$
\end{itemize}

\bigskip
\noindent{\bf 20.}\upabit
\begin{itemize}
  \item[{a)}] By Problem {\bf 20b}, we know that $d$ is the minimum number of dependent columns of $H$.\\
     Since two columns of $H$ are parallel (identical), $d\geq 3$.\\
     In fact, the first three columns of $H$ are dependent,
     so $d = 3$.
  \item[{b)}] Writing $d = 2t + 1$, we see that $H$ can detect and correct $t = 1$ error.
  \item[{c)}]
    \begin{itemize}
      \item[(i)]   $\vc{1 &1 &1 &1 &0 &0\\
                        0 &1 &1 &0 &1 &0\\
                        1 &0 &1 &0 &0 &1}
                    \vc{0\\1\\0\\1\\1\\0}
                  = \vc{0\\0\\0}
                  = \vec{0}$
                  \\We see that 010110 is a codeword and does not need correcting.
                    Decoding gives 010.
      \item[(ii)]  $\vc{1 &1 &1 &1 &0 &0\\
                        0 &1 &1 &0 &1 &0\\
                        1 &0 &1 &0 &0 &1}
                    \vc{0\\1\\0\\0\\0\\1}
                  = \vc{1\\1\\1}$
                  \\This is the 3rd column of $H$,
                    so the 3rd bit is incorrect (assuming just a single bit-error):\\
                    the correct codeword is then 011001.
                    Decoding gives 011.
      \item[(iii)]$\vc{1 &1 &1 &1 &0 &0\\
                       0 &1 &1 &0 &1 &0\\
                       1 &0 &1 &0 &0 &1}
                   \vc{1\\0\\0\\1\\1\\0}
                 = \vc{0\\1\\1}$
                  \\This is not one of the columns of $H$,
                    so we see that 100110 has at least two errors.
                    However, we cannot determine which they might be:
                    for instance, the syndrome could be the sum of columns 1 and 2 of $H$ -
                    or it could be the sum of columns 5 and 6, say.
     \end{itemize}
  \item[{d)}] Row-reduce $H$:
                  $\vc{1 &1 &1 &1 &0 &0\\
                       0 &1 &1 &0 &1 &0\\
                       1 &0 &1 &0 &0 &1}
%                   \ra{R_3 = R_3 + R_1}
%                   \vc{1 &1 &1 &1 &0 &0\\
%                       0 &1 &1 &0 &1 &0\\
%                       0 &1 &0 &1 &0 &1}
%                   \dra{R_1 = R_2 + R_1}{R_3 = R_2 + R_3}
%                   \vc{1 &0 &0 &1 &1 &0\\
%                       0 &1 &1 &0 &1 &0\\
%                       0 &0 &1 &1 &1 &1}
%                   \ra{R_2 = R_2 + R_3}
                   \longrightarrow\cdots\longrightarrow
                   \vc{1 &0 &0 &1 &1 &0\\
                       0 &1 &0 &1 &0 &1\\
                       0 &0 &1 &1 &1 &1} = H'$
                   \\[2mm]Then $G' =
                   \vc{1 &1 &1 & 1 &0 &0\\
                       1 &0 &1 & 0 &1 &0\\
                       0 &1 &1 & 0 &0 &1}$.
                   \\[2mm]This also serves as a generator matrix for $H$.
  \item[{e)}] Let $H^+ =
                   \vc{1 &1 &1 &1 &1 &1 &1\\
                       0 &1 &1 &1 &1 &0 &0\\
                       0 &0 &1 &1 &0 &1 &0\\
                       0 &1 &0 &1 &0 &0 &1}$.
              This is just $H$ with an added $\vec{0}$-column and an added $\vec{1}$-row.
              There is no zero column or any two parallel (identical) columns;
              hence, the minimal distance $d^+$ of the code $C^+$ defined by $H^+$ is
              at least 3. Furthermore, there are no three columns of $H^+$ that
              are linearly independent since in $\mathbb{Z}_2$, this mean that
              one of the three vectors were the sum of the other two - which cannot
              happen here, since the first coordinates of the columns all equal~1.
              Hence, $d^+ \geq 4$. Since $d^+\leq d + 1 = 4$, we see that
              the extended code $C^+$ has minimum distance $d^+ = 4$.
\end{itemize}

\bigskip
\noindent{\bf 21.}\upabit
\begin{itemize}
  \item[{a)}] We wish to find a code $C$ with at least $|C|\geq 4$ codewords, of length $n$ say,
              and minimum distance $d \geq 2t+1 = 3$ for $t=1$ (the number of errors that we want to correct).\\
              The Sphere Packing Bound gives $|C|\sum_{i=0}^t \leq 2^n$; i.e., $4(1+n)\leq 2^n$.\\
              We can quickly check that this is not true for $n = 1,\ldots,4$,
              so we must have that $n\geq 5$.\\
              The following parity check matrix defines a code with length $n=5$ and $d=3$:
            \[H = \vc{1 &0 &0 &1 &1\\
                      0 &1 &0 &1 &1\\
                      0 &0 &1 &1 &0}\]
              Alternatively, the following generator matrix defines a (different) code $C$ with length $n = 5$ and $d = 3$:
            \[G = \vc{1 &1 &1 &0 &0\\
                      0 &0 &1 &1 &1}\]
             \\It has just 4 codewords, which is perfect for our purposes.
  \item[{b)}] We wish to find a code $C$ with at least $|C|\geq 4$ codewords, of length $n$ say,
              and minimum distance $d \geq 2t+1 = 5$ for $t=2$ (the number of errors that we want to correct).\\
              The Sphere Packing Bound gives $|C|\sum_{i=0}^t \leq 2^n$; i.e., $4(1+n+\binom{n}{2})\leq 2^n$.\\
              We can quickly check that this is not true for $n = 1,\ldots,6$,
              so we must have that $n\geq 7$.\\
              The following generator matrix defines a code $C$ with length $n = 8$ and $d = 5$:
            \[G = \vc{1 &1 &1 &1 &1 &0 & 0& 0\\
                      0 &0 &0 &1 &1 &1 & 1& 1}\]
             \\It has just 4 codewords, which is perfect for our purposes.
\end{itemize}

\bigskip
\noindent{\bf 22.}\upabit
\begin{itemize}
  \item[{a)}] Choose the $\rho$ differing coordinates in $\binom{n}{\rho}$ ways;
              for each of the $\rho$ coordinates,
              there are $r-1$ symbols that replace the present one.
              All in all, there are than $\binom{n}{\rho}(r-1)^\rho$ vectors in $\mathbb{Z}_r^n$
              at distance $\rho$ from $\vec{x}$.
  \item[{b)}] $\ds|C|\sum_{i=0}^t \binom{n}{i}(r-1)^i \leq r^n$.
  \item[{c)}] For each radix $r$ Hamming code $C$, $t=1$ and $|C| = r^{n-k}$ for some $k$
              where
              \[
                n  = (r^k - 1)/(r-1) = \sum_{j=0}^{k-1} r^j
              \]
              is the length of $C$.
              (The columns of $H$ are all the vectors of $\mathbb{Z}_r$, except $\vec{0}$,
              and except that each set of $r-1$ parallel vectors is replaced by just a single vector.)
              Therefore,
              \[
                  |C|\sum_{i=0}^t \binom{n}{i}(r-1)^i
                = r^{n-k} (1 + n(r-1))
                = r^{n-k} (1 + (r^k-1))
                = r^n\,.
              \]
\end{itemize}

\bigskip
\noindent{\bf 23.}\upabit
\begin{itemize}
  \item[{a)}] $H = \vc{1 & 0 & 1 & 1 & 1 & 1\\
                       0 & 1 & 1 & 2 & 3 & 4}$
  \item[{b)}] $H\vec{y} = \vc{1 & 0 & 1 & 1 & 1 & 1\\
                              0 & 1 & 1 & 2 & 3 & 4}
                          \vc{4\\ 1\\ 0\\ 0\\ 1\\ 3}
                        = \vc{8\\16}
                        = \vc{3\\1}$
            \\This is 3 times the 4th column, so (assuming a single error)
            the 2nd entry of $\vec{y}$ is 3 too big;
            subtracting 3 then gives the corrected codeword is then 410213.
            Decoding then gives the message 0213.
\end{itemize}

\bigskip
\noindent{\bf 24.}\upabit
\begin{itemize}
  \item[{a)}] Setting $x_3 = 2$ and $x_4 = 1$ gives us
    \begin{align*}
      x_1 +  x_2 + 3\times 2 + 2\times 1 \equiv 0\pmod{5}\\
      x_1 + 2x_2 + 4\times 2 + 3\times 1 \equiv 0\pmod{5}
    \end{align*}
    or, in other words,
    \begin{align*}
      x_1 +  x_2 \equiv 2\pmod{5}\\
      x_1 + 2x_2 \equiv 4\pmod{5}
    \end{align*}
    Quickly solving this gives us $x_1 = 0$ and $x_2 = 2$,
    so 21 encodes as 0221.
    \\(Check: this is indeed a codeword.)
  \item[{b)}] Just check the two congruences:\\
              (1) Invalid \: (2) Valid \: (3) Invalid \: (4) Valid
\end{itemize}

\newpage
\noindent{\bf 25.}\upabit
\begin{itemize}
  \item[{a)}] Neither (eg., $0 + 0 = 00$).
  \item[{b)}] Not instantaneous (eg., 0 is a prefix of 01) but UD.
  \item[{c)}] Neither (eg., $001 + 0 = 0010$)
  \item[{d)}] Instantaneous.
\end{itemize}

\bigskip
\noindent{\bf 26.}\quad
The code is not UD; for instance, 
 $\vec{c}_4\vec{c}_6\vec{c}_2\vec{c}_3
= \vec{c}_7\vec{c}_1\vec{c}_5\vec{c}_6$:
\[
    \underbrace{1110}_{\vec{c}_4}\underbrace{010100}_{\vec{c}_6}\underbrace{0011}_{\vec{c}_2}\underbrace{1001}_{\vec{c}_3}
  = \underbrace{11100}_{\vec{c}_7}\underbrace{101}_{\vec{c}_1}\underbrace{00001}_{\vec{c}_5}\underbrace{11001}_{\vec{c}_6}\,.
\]


\end{document}
