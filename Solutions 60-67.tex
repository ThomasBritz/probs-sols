\documentclass[11pt]{article}
\usepackage{amsmath,amssymb,latexsym}
%\usepackage{amsmath,amssymb,latexsym,pstricks}

\parindent 0in
\addtolength{\textwidth}{52mm}
\addtolength{\oddsidemargin}{-28mm}
\addtolength{\evensidemargin}{-26mm}
\addtolength{\topmargin}{-20mm}
\addtolength{\textheight}{40mm}

\newcommand{\ph}{\phantom}
\newcommand{\ds}{\displaystyle}
\newcommand{\tvs}{\textvisiblespace}
\newcommand{\mymod}[1]{\,(\textrm{mod}\,#1)}
\renewcommand{\vec}[1]{\mathbf{#1}}
\newcommand{\vc}[1]{\begin{pmatrix}#1\end{pmatrix}}
\newcommand{\tmat}[1]{{\left(\begin{array}{rrr}#1\end{array}\right)}}
\newcommand{\qmat}[1]{{\left(\begin{array}{rrrr}#1\end{array}\right)}}
\newcommand{\pmat}[1]{{\left(\begin{array}{rrrrr}#1\end{array}\right)}}
\newcommand{\proof}{{\sc Proof.}\quad}
\newcommand{\qed}{\quad$\square$}

%\newcommand{\myup}{\begin{picture}(0,0)(0,20)\ph{bla}\end{picture}}
\newcommand{\moveup}{\begin{picture}(0,0)(0,0)\end{picture}\vspace*{-8.15mm}}
\newcommand{\upabit}{\begin{picture}(0,0)(0,0)\end{picture}\vspace*{-5mm}}

\newcommand{\ra}[1]{\xrightarrow{#1}}
\newcommand{\dra}[2]{\begin{array}{c}\xrightarrow{\text{\scriptsize$#1$}}\\[-2mm]{\text{\scriptsize$#2$}}\end{array}}


\date{}
\author{}
\title{\sc Solutions to MATH3411 Problems 60--67}


\begin{document} \maketitle

\vspace*{-10mm}

\noindent{\bf 60.}
\begin{itemize}
  \item[{(a)}] $\phi(17) = 16$ and $\gcd(2,17)=1$, so by Euler's Theorem, $2^{16}\equiv1\pmod{17}$, so
    \[2^{1001} = 2^{16\times 62+9} = \bigl(2^{16}\bigr)^{62}(2^4)^2 2 \equiv 1^{62}16^2\times2\equiv(-1)^2 2\equiv 2\pmod{17}\]
  \item[{(a)}] $\phi(100) = \phi(2^2)\phi(5^2) = 2\times 20 = 40$ and $\gcd(3,100)=1$, so by Euler's Theorem, $3^{40}\equiv1\pmod{100}$, so
    \[3^{1001} = 3^{40\times 25+1} = \bigl(3^{40}\bigr)^{25}3 \equiv 1^{25}3\equiv 3\pmod{100}\]
    so the two last digits are ``03".
\end{itemize}

\bigskip\noindent{\bf 61.}
First try to find a primitive root for $\mathbb{Z}_{11}$,
first trying 2: its powers $2^1,\ldots,2^{10}$ are
\[
  2, 4, 8, 5, 10, 9, 7, 3, 6, 1
\]
and we see that 2 is a primitive root for $\mathbb{Z}_{11}$.
(Here, we could reduce calculations by checking
that $2^\frac{11-1}{5} = 2^2 \neq 1$
and  $2^\frac{11-1}{2} = 2^5 \neq 1$,
since the order of $\mathbb{U}_{11}$ is $10 = 2\times 5$ which has prime factors~2 and~5.)
All ($\phi(\phi(11)) = 4$) primitive roots for $\mathbb{Z}_{11}$ are now given by $2^i$ with $\gcd(i,11-1) = \gcd(i,10) = 1$;
that is
\[
  2^1 = 1, \quad 2^3 = 8, \quad 2^7 = 7, \quad 2^9 = 6
\]
Now try to find a primitive root for $\mathbb{Z}_{17}$,
first trying 2: its powers $2^1,\ldots,2^{16}$ are
\[
  2, 4, 8, 16, 15, 13, 9, 1,\ldots
\]
We see that 2 is a not primitive root for $\mathbb{Z}_{17}$.
(Here, it is enough just to test whether $2^\frac{16}{2} = 2^8 = 1$ or not,
since the order of $\mathbb{U}_{17}$ is $16 = 2^4$ which just contains the single prime factor~2.)\\
Let us then try 3: its powers $3^1,\ldots,3^{16}$ are
\[
  3, 9, 10, 13, 5, 15, 11, 16, 14, 8, 7, 4, 12, 2, 6, 1
\]
We see that 3 is a primitive root for $\mathbb{Z}_{17}$.
(Again, we really just needed just check that $3^\frac{17-1}{2} = 3^8 \neq 1$.)
All ($\phi(\phi(11)) = 4$) primitive roots for $\mathbb{Z}_{17}$
are now given by $3^i$ with $\gcd(i,17-1) = \gcd(i,16) = 1$;\\
that is the odd powers of 3:
\[
  3, 10, 5, 11, 14, 7, 12, 6
\]

\newpage\noindent{\bf 62.}
\begin{itemize}
  \item[{(a)}]
    First use the Euclidean algorithm forwards (in $\mathbb{Z}$):
    \begin{align*}
      x^3+1 &=     x \times (x^2+1) + (-x+1)\\
      x^2+1 &= (-x-1)\times  (-x+1) +  2\\
       -x+1 &= \frac{1}{2}(-x+1)\times 2 + 0
    \end{align*}
    so a greatest common divisor of $f = x^3 + 1$ and $g = x^2 + 1$ is 2.\\
    Scaling to get a monic polynomial gives us that $d = \gcd(f,g) = 1$.\\
    Now use the Euclidean algorithm backwards, letting $h = -x + 1$:
    \begin{align*}
       2  &= g - (-x-1)h\\
          &= g - (-x-1)(f - xg)\\
          &= (x+1)f + (-x^2 - x + 1) g\\
    \end{align*}
    Hence, $d = \gcd(f,g) = 1 = af + bg$
    for $a = \frac{1}{2}(x+1)$
    and $b = \frac{1}{2}(-x^2 - x + 1)$.
  \item[{(b)}]
    First use the Euclidean algorithm forwards (in $\mathbb{Z}_2$):
    \begin{align*}
      x^3+1 &=     x \times (x^2+1) + (x+1)\\
      x^2+1 &= (x+1) \times  (x+1) +  0\\
    \end{align*}
    so $d = \gcd(f,g) = x + 1$.\\
    Now use the Euclidean algorithm backwards:
    \[
      x + 1 = f - xg
    \]
    Hence, $d = \gcd(f,g) = x + 1 = af + bg$ for $a = 1$ and $b = -x$.
  \item[{(c)}]
    First use the Euclidean algorithm forwards (in $\mathbb{Z}_3$):
    \begin{align*}
      x^3-x^2-1 &=   x\times (x^2 - x + 1) + (-x-1)\\
      x^2-x + 1 &= (-x+2)\times (-x-1) +  0
    \end{align*}
    so a greatest common divisor of $f = x^3 + 1$ and $g = x^2 + 1$ is $-x-1$.\\
    Scaling to get a monic polynomial gives us that $d = \gcd(f,g) = x+1$.\\
    Now use the Euclidean algorithm backwards:
    \[
      -x-1 = f - xg
    \]
    Hence, $d = \gcd(f,g) = x+1 = af + bg$
    for $a = -1$
    and $b = x$.
\end{itemize}

\newpage\noindent{\bf 63.}
\begin{itemize}
  \item[{(a)}] In $\mathbb{Z}_2$ and modulo $x^2 + x + 1$,
    \begin{align*}
         x^5 + x^2 + 1
      &= x^5 + x^2 + 1 + x^3(x^2+x+1)\\
      &= x^4 + x^3 + x^2 + 1\\
      &= x^4 + x^3 + x^2 + 1 + x^2(x^2+x+1)\\
      &= 1
    \end{align*}
  \item[{(b)}] In $\mathbb{Z}_3$ and modulo $x^2 + x + 1$,
    \begin{align*}
         x^5 + x^2 + 1
      &= x^5 + x^2 + 1 - x^3(x^2+x+1)\\
      &=-x^4 - x^3 + x^2 + 1\\
      &=-x^4 - x^3 + x^2 + 1 + x^2(x^2+x+1)\\
      &= 2x^2 + 1\\
      &= 2x^2 + 1 + (x^2 + x + 1)\\
      &= x + 2
    \end{align*}
\end{itemize}


\noindent{\bf 64.}
\[\begin{array}{c}\begin{array}{|c|cccc|}\hline
    +       &       0 &       1 &     x   &     x+1\\\hline
          0 &       0 &       1 &     x   &     x+1\\
          1 &       1 &       0 &     x+1 &     x  \\
        x   &     x   &     x+1 &       0 &       1\\
        x+1 &     x+1 &     x   &       1 &       0\\\hline
  \end{array}\\[14mm]\textrm{both}\end{array}\quad
  \begin{array}{c}\begin{array}{|c|cccc|}\hline
   \times   &       0 &       1 &     x   &     x+1\\\hline
          0 &       0 &       0 &       0 &       0\\
          1 &       0 &       1 &     x   &     x+1\\
        x   &       0 &     x   &     x+1 &       1\\
        x+1 &       0 &     x+1 &       1 &     x  \\\hline
  \end{array}\\[14mm]\mathbb{Z}_2/\langle x^2+x+1\rangle\end{array}\quad
  \begin{array}{c}\begin{array}{|c|cccc|}\hline
   \times   &       0 &       1 &     x   &     x+1\\\hline
          0 &       0 &       0 &       0 &       0\\
          1 &       0 &       1 &     x   &     x+1\\
        x   &       0 &     x   &       1 &     x+1\\
        x+1 &       0 &     x+1 &     x+1 &       0\\\hline
  \end{array}\\[14mm]\mathbb{Z}_2/\langle x^2 + 1\rangle\end{array}\]
Every non-zero element in $\mathbb{Z}_2/\langle x^2+x+1\rangle$ has an inverse and is therefore a unit,
\\in contrast to the element $x+1$ in $\mathbb{Z}_2/\langle x^2+1\rangle$.
\\Therefore,  $\mathbb{Z}_2/\langle x^2+x+1\rangle$ is a field and $\mathbb{Z}_2/\langle x^2+1\rangle$ is not.

\bigskip\noindent{\bf 65.}
Since $m = x^4+x^2+x+1$ has 1 as a root in $\mathbb{Z}_2$,
it is divisible by $x-1$ and is therefore not irreducible.\\
Hence, $\mathbb{Z}_2/\langle x^4+x^2+x+1\rangle$ is not a field.

\medskip
Let us see whether $m$ is irreducible in $\mathbb{Z}_3$.\\
We first note that $m(0) = m(1) = 1$ and $m(2) = 2$, so $m$ has no roots and thus no linear factor.\\
Suppose that
\[
  m =  x^4 + x^2+ x + 1
    = (x^2 + ax + b)(x^2 + cx + d)
    =  x^4 + (a+c)x^3 + (b+ac+d)x^2 + (ad+bc)x + bd
\]
Comparing terms, we see that $c = -a$ and that $b = d^{-1} = d\neq 0$ (in $\mathbb{Z}_3$).\\
Therefore, $ad+bc = ad-da = 0\neq 1$, a contradiction.\\
Therefore, $m$ has no linear or quadratic divisors in $\mathbb{Z}_3$ and must be irreducible.\\
Hence, $\mathbb{Z}_2/\langle x^4+x^2+x+1\rangle$ is a field.

\newpage\noindent{\bf 66.}
\begin{itemize}
  \item[{(a)}] Here, we have that $\alpha^3 = -\alpha-1 = \alpha+1$.
  \[\begin{array}{|c||c|c|c|c|c|c|c|c|}\hline
           i &  0  &    1   &    2     &    3     &    4            &             5     &    6       &  7\\\hline
    \alpha^i &  1  & \alpha & \alpha^2 & \alpha+1 & \alpha^2+\alpha & \alpha^2+\alpha+1 & \alpha^2+1 &  1\\\hline
  \end{array}\]
    The element $\alpha$ is primitive, so all of the primitive elements of $\mathbb{Z}_2/\langle x^3+x+1\rangle$
    are given by $\alpha^i$ where $\gcd(i,7) = 1$; that is all of the 6 elements listed above: $\alpha,\ldots,\alpha^2+1$.
  \item[{(b)}] Here, $\alpha^4 = \alpha^3+\alpha^2+\alpha+1$.
  \\Since $\alpha^5 = 1$, $\text{ord}(\alpha)\leq 5<8$, so $\alpha$ is not primitive in $\mathbb{Z}_2/\langle x^4+x^3+x^2+x+1\rangle$.
  \\Let us therefore consider $\gamma = \alpha + 1$ for instance:
  \[\begin{array}{|ll|ll|}\hline\begin{picture}(0,0)(0,.7)\end{picture}
    \gamma^0 \!\!\!&= 1                                               & \gamma^8   \!\!\!&= \gamma^3 + \gamma^2 + \gamma = \alpha^3 + 1 \\
    \gamma^1 \!\!\!&= \alpha + 1                                      & \gamma^9   \!\!\!&= \gamma^2 + 1 = \alpha^2 \\
    \gamma^2 \!\!\!&= \alpha^2+1                                      & \gamma^{10}\!\!\!&= \gamma^3 + \gamma = \alpha^3 + \alpha^2 \\
    \gamma^3 \!\!\!&= \alpha^3 + \alpha^2 + \alpha + 1                & \gamma^{11}\!\!\!&= \gamma^3 + \gamma^2 + 1 = \alpha^3 + \alpha + 1\\
    \gamma^4 \!\!\!&= \alpha^3 + \alpha^2 + \alpha = \gamma^3 + 1     & \gamma^{12}\!\!\!&= \gamma + 1 = \alpha \\
    \gamma^5 \!\!\!&= \gamma^3 + \gamma + 1 = \alpha^3 + \alpha^2 + 1 & \gamma^{13}\!\!\!&= \gamma^2 + \gamma = \alpha^2 + \alpha \\
    \gamma^6 \!\!\!&= \gamma^3 + \gamma^2 + \gamma + 1 = \alpha^3     & \gamma^{14}\!\!\!&= \gamma^3 + \gamma^2 = \alpha^3 + \alpha\\
    \gamma^7 \!\!\!&= \gamma^2 + \gamma + 1 = \alpha^2 + \alpha + 1   & \gamma^{15}\!\!\!&= 1\\\hline
  \end{array}\]
    The element $\gamma$ is primitive, so all of the primitive elements of $\mathbb{Z}_2/\langle x^4+x^3+x^2+1\rangle$
    are given by $\alpha^i$ where $\gcd(i,15) = 1$; that is the $\phi(15) = 8$ elements
    {\small\[
      \gamma   = \alpha + 1,\;
      \gamma^2 = \alpha^2 + 1,\;
      \gamma^4 = \alpha^3 + \alpha^2 + \alpha,\;
      \gamma^7 = \alpha^2 + \alpha + 1,\;
      \gamma^8 = \alpha^3 + 1,\;
      \gamma^{11} = \alpha^3 + \alpha + 1,\;\\
      \gamma^{13} = \alpha^2 + \alpha,\;\\
      \gamma^{14} = \alpha^3 + \alpha
    \]}
  \item[{(c)}] Here, $\alpha^2 = -\alpha + 1 = 2\alpha + 1$.
  \[\begin{array}{|c||c|c|c|c|c|c|c|c|c|}\hline
           i &  0  &    1   &    2      &     3     & 4 &    5    &     6      &     7      & 8 \\\hline
    \alpha^i &  1  & \alpha & 2\alpha+1 & 2\alpha+2 & 2 & 2\alpha & \alpha + 2 & \alpha + 1 & 1 \\\hline
  \end{array}\]
    The element $\alpha$ is primitive, so all of the primitive elements of $\mathbb{Z}_3/\langle x^2+x-1\rangle$
    are given by $\alpha^i$ where $\gcd(i,8) = 1$; that is all of odd powers of $\alpha$ above: 
    \[
      \alpha\,,\quad
      \alpha^3 = 2\alpha + 2\,,\quad
      \alpha^5 = 2\alpha\,,\quad
      \alpha^7 =  \alpha + 1
    \]
\end{itemize}

\newpage\noindent{\bf 67.}
  Here, we have that $\alpha^4 = \alpha + 1$.
  \[\begin{array}{|ll|ll|}\hline\begin{picture}(0,0)(0,.7)\end{picture}
    \alpha^0 \!\!\!&= 1                     & \alpha^8   \!\!\!&= \alpha^2 + 1 \\
    \alpha^1 \!\!\!&= \alpha                & \alpha^9   \!\!\!&= \alpha^3 + \alpha \\
    \alpha^2 \!\!\!&= \alpha^2              & \alpha^{10}\!\!\!&= \alpha^2 + \alpha + 1\\
    \alpha^3 \!\!\!&= \alpha^3              & \alpha^{11}\!\!\!&= \alpha^3 + \alpha^2 + \alpha\\
    \alpha^4 \!\!\!&= \alpha + 1            & \alpha^{12}\!\!\!&= \alpha^3 + \alpha^2 + \alpha + 1 \\
    \alpha^5 \!\!\!&= \alpha^2 + \alpha     & \alpha^{13}\!\!\!&= \alpha^3 + \alpha^2 + 1 \\
    \alpha^6 \!\!\!&= \alpha^3 + \alpha^2   & \alpha^{14}\!\!\!&= \alpha^3 + 1\\
    \alpha^7 \!\!\!&= \alpha^3 + \alpha + 1 & \alpha^{15}\!\!\!&= 1\\\hline
  \end{array}\]
  Define $\beta = \alpha^3+\alpha+1$.
\begin{itemize}
  \item[{(a)}] By the table, we see that $\beta = \alpha^7$, so 
    \[
      (\alpha^3+\alpha+1)^{-1} = \beta^{-1} = \alpha^{-7} = \alpha^{15-7} = \alpha^8 = \alpha^2 + 1
    \]
  \item[{(b)}] By the table, 
    \[
        \frac{(\alpha^3+\alpha+1)(\alpha+1)}{\alpha^3+1} 
      = \frac{\alpha^7\alpha^4}{\alpha^{14}} 
      = \alpha^{-3}
      = \alpha^{15-3}
      = \alpha^{12}
      = \alpha^3 + \alpha^2 + \alpha + 1 
    \]
  \item[{(c)}] For $p = 2$, the powers $\beta^{p^i} = \alpha^{7\times 2^i}$ are
  \[
    \alpha^7\,,\quad
    \alpha^{14}\,,\quad
    \alpha^{28} = \alpha^{13}\,,\quad
    \alpha^{56} = \alpha^{11}\,,\quad
    \alpha^{112} = \alpha^{7}\,,
  \]
  so the minimal polynomial of $\beta = \alpha^3+\alpha+1 = \alpha^7$ is
    \begin{align*}
      g(x) &= (x-\alpha^7)(x-\alpha^{14})(x-\alpha^{13})(x-\alpha^{11})\\
           &= x^4 - (\alpha^7+\alpha^{14}+\alpha^{13}+\alpha^{11})x^3\\
           &\ph{=} + (\alpha^7\alpha^{14}+\alpha^7\alpha^{13}+\alpha^7\alpha^{11}+\alpha^{14}\alpha^{13}+\alpha^{14}\alpha^{11}+\alpha^{13}\alpha^{11})x^2\\
           &\ph{=} - (\alpha^7\alpha^{14}\alpha^{13}+\alpha^7\alpha^{14}\alpha^{11}+\alpha^7\alpha^{13}\alpha^{11}+\alpha^{14}\alpha^{13}\alpha^{11})x\\
           &\ph{=} +  \alpha^7\alpha^{14}\alpha^{13}\alpha^{11}\\
           &= x^4 - (\alpha^7+\alpha^{14}+\alpha^{13}+\alpha^{11})x^3\\
           &\ph{=} + (\alpha^6+\alpha^5+\alpha^3+\alpha^{12}+\alpha^{10}+\alpha^9)x^2\\
           &\ph{=} - (\alpha^3+\alpha^2+\alpha+\alpha^8)x\\
           &\ph{=} +  \alpha^0\\
           &= x^4 - ((\alpha^3 + \alpha + 1) +(\alpha^3 + 1) + (\alpha^3 + \alpha^2 + 1) + (\alpha^3 + \alpha^2 + \alpha)x^3\\
           &\ph{=} + ((\alpha^3 + \alpha^2) + (\alpha^2 + \alpha) + \alpha^3 + (\alpha^3 + \alpha^2 + \alpha + 1) + (\alpha^2 + \alpha + 1) + (\alpha^3 + \alpha))x^2\\
           &\ph{=} - (\alpha^3+\alpha^2+\alpha+\alpha^2 + 1)x\\
           &\ph{=} + 1\\
           &= x^4 - x^3 - (\alpha^3+\alpha+ 1)x + 1\\
           &= x^4 + x^3 + 1
    \end{align*} 
  \item[{(d)}] The (primitive) elements $\alpha^1 = \alpha$, $\alpha^2$, $\alpha^4$, and $\alpha^8$ have the (primitive) minimum polynomial $x^4 + x + 1$.\\
    The powers $(\alpha^3)^{2^i}$ are
    \[
      \alpha^3\,,\quad
      \alpha^6\,,\quad
      \alpha^{12}\,,\quad
      \alpha^{24} = \alpha^9\,,\quad
      \alpha^{48} = \alpha^{3}\,,
    \]
    so the minimal polynomial of $\alpha^3$ is
    \begin{align*}
      g(x) &= (x-\alpha^3)(x-\alpha^6)(x-\alpha^{12})(x-\alpha^9)\\
           &= x^4 - (\alpha^3+\alpha^6+\alpha^{12}+\alpha^9)x^3\\
           &\ph{=} + (\alpha^3\alpha^6+\alpha^3\alpha^{12}+\alpha^3\alpha^9+\alpha^6\alpha^{12}+\alpha^6\alpha^9+\alpha^{12}\alpha^9)x^2\\
           &\ph{=} - (\alpha^3\alpha^6\alpha^{12}+\alpha^3\alpha^6\alpha^9+\alpha^3\alpha^{12}\alpha^9+\alpha^6\alpha^{12}\alpha^9)x\\
           &\ph{=} +  \alpha^3\alpha^6\alpha^{12}\alpha^9\\
           &= x^4 - (\alpha^3+\alpha^6+\alpha^{12}+\alpha^9)x^3\\
           &\ph{=} + (\alpha^9+\alpha^0+\alpha^{12}+\alpha^3+\alpha^0+\alpha^6)x^2\\
           &\ph{=} - (\alpha^6+\alpha^3+\alpha^9+\alpha^{12})x\\
           &\ph{=} +  \alpha^0\\
           &= x^4 - (\alpha^3 + (\alpha^3 + \alpha^2) + (\alpha^3 + \alpha^2 + \alpha + 1) + (\alpha^3 + \alpha))x^3\\
           &\ph{=} + ((\alpha^3 + \alpha) + 1 + (\alpha^3 + \alpha^2 + \alpha + 1) + \alpha^3 + 1 + (\alpha^3 + \alpha^2))x^2\\
           &\ph{=} - ((\alpha^3 + \alpha^2) + \alpha^3 + (\alpha^3 + \alpha) + (\alpha^3 + \alpha^2 + \alpha + 1))x\\
           &\ph{=} +  1\\
           &= x^4 + x^3 + x + 1
    \end{align*}
    The powers $(\alpha^5)^{2^i}$ are
    \[
      \alpha^5\,,\quad
      \alpha^{10}\,,\quad
      \alpha^{20} = \alpha^5\,,
    \]
    so the minimal polynomial of $\alpha^5$ is
    \begin{align*}
      g(x) &= (x-\alpha^5)(x-\alpha^{10})\\
           &= x^2 - (\alpha^5+\alpha^{10})x + \alpha^5\alpha^{10}\\
           &= x^2 - ((\alpha^2+\alpha) + (\alpha^2 + \alpha + 1))x + \alpha^0\\
           &= x^2 + x + 1
    \end{align*}
    The powers $(\alpha^5)^{2^i}$ are
    \[
      \alpha^5\,,\quad
      \alpha^{10}\,,\quad
      \alpha^{20} = \alpha^5\,,
    \]
    so the minimal polynomial of $\alpha^5$ is
    \begin{align*}
      g(x) &= (x-\alpha^5)(x-\alpha^{10})\\
           &= x^2 - (\alpha^5+\alpha^{10})x + \alpha^5\alpha^{10}\\
           &= x^2 - ((\alpha^2+\alpha) + (\alpha^2 + \alpha + 1))x + \alpha^0\\
           &= x^2 + x + 1
    \end{align*}
    We have thus found the minimal polynomials of the powers of $\alpha$:
    \[
      x^4 + x^3 + 1\,,\quad x^4 + x^3 + x + 1\,,\quad x^2 + x + 1
    \]
\end{itemize}

\end{document}

