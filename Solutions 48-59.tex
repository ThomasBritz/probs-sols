\documentclass[11pt]{article}
\usepackage{amsmath,amssymb,latexsym}
%\usepackage{amsmath,amssymb,latexsym,pstricks}

\parindent 0in
\addtolength{\textwidth}{52mm}
\addtolength{\oddsidemargin}{-28mm}
\addtolength{\evensidemargin}{-26mm}
\addtolength{\topmargin}{-20mm}
\addtolength{\textheight}{40mm}

\newcommand{\ph}{\phantom}
\newcommand{\ds}{\displaystyle}
\newcommand{\tvs}{\textvisiblespace}
\newcommand{\mymod}[1]{\,(\textrm{mod}\,#1)}
\renewcommand{\vec}[1]{\mathbf{#1}}
\newcommand{\vc}[1]{\begin{pmatrix}#1\end{pmatrix}}
\newcommand{\tmat}[1]{{\left(\begin{array}{rrr}#1\end{array}\right)}}
\newcommand{\qmat}[1]{{\left(\begin{array}{rrrr}#1\end{array}\right)}}
\newcommand{\pmat}[1]{{\left(\begin{array}{rrrrr}#1\end{array}\right)}}
\newcommand{\proof}{{\sc Proof.}\quad}
\newcommand{\qed}{\quad$\square$}

\newcommand{\moveup}{\begin{picture}(0,0)(0,0)\end{picture}\vspace*{-8.15mm}}
\newcommand{\upabit}{\begin{picture}(0,0)(0,0)\end{picture}\vspace*{-5mm}}

\newcommand{\ra}[1]{\xrightarrow{#1}}
\newcommand{\dra}[2]{\begin{array}{c}\xrightarrow{\text{\scriptsize$#1$}}\\[-2mm]{\text{\scriptsize$#2$}}\end{array}}


\date{}
\author{}
\title{\sc Solutions to MATH3411 Problems 48--59}


\begin{document} \maketitle

\vspace*{-10mm}

\noindent{\bf 48.}
Define $A = \{a_1,a_2\}$, $B = \{b_1,b_2\}$, $C = \{c_1,c_2\}$, where
\begin{center}
\begin{tabular}{lll}
  $a_1$: student passes\\
  $a_2$: student fails\\
  $b_1$: student owns car\\
  $b_2$: student owns no car\\
  $c_1$: student lives at home\\
  $c_2$: student lives away from home
\end{tabular}
\end{center}
We are told, and can immediately infer, that
  \[
    \begin{array}{lllclllclllclll}
             & &       & \qquad& P(b_1|a_1) &=& 0.10 & \qquad &            & &      &       & P(c_1|b_2\cap a_1) &=& 0.40\\
      P(a_1) &=& 0.75  &       & P(b_2|a_1) &=& 0.90 &        & P(c_1|b_1) &=& 1.00 &\qquad & P(c_2|b_2\cap a_1) &=& 0.60\\
      P(a_2) &=& 0.25  &       & P(b_1|a_2) &=& 0.50 &        & P(c_2|b_1) &=& 0    &       & P(c_1|b_2\cap a_2) &=& 0.40\\
             & &       &       & P(b_2|a_2) &=& 0.50 &        &            & &      &       & P(c_2|b_2\cap a_2) &=& 0.60
    \end{array}
  \]
From this, we can calculate
  \[
    \begin{array}{lllclllclllclll}
      P(a_1\cap b_1) &=& 0.075 & \qquad&        & &     & \qquad &                & &     &       & P(a_1\cap b_2\cap c_1) &=& 0.27 \\
      P(a_1\cap b_2) &=& 0.675 &       & P(b_1) &=& 0.2 &        & P(b_1\cap c_1) &=& 0.2 &\qquad & P(a_1\cap b_2\cap c_2) &=& 0.405\\
      P(a_2\cap b_1) &=& 0.125 &       & P(b_2) &=& 0.8 &        & P(b_1\cap c_2) &=& 0   &       & P(a_2\cap b_2\cap c_1) &=& 0.05 \\
      P(a_2\cap b_2) &=& 0.125 &       &        & &     &        &                & &     &       & P(a_2\cap b_2\cap c_2) &=& 0.075
    \end{array}
  \]
and thus
  \[
    \begin{array}{lllclllclll}
                     & &      & \qquad&        & &      & \qquad & P(a_1\cap b_1\cap c_1) &=& 0.075\\
      P(b_2\cap c_1) &=& 0.32 & \qquad& P(c_1) &=& 0.52 & \qquad & P(a_1\cap b_1\cap c_2) &=& 0\\
      P(b_2\cap c_2) &=& 0.48 &       & P(c_2) &=& 0.48 & \qquad & P(a_2\cap b_1\cap c_1) &=& 0.125\\
                     & &      & \qquad&        & &      & \qquad & P(a_2\cap b_1\cap c_2) &=& 0
    \end{array}
  \]
and so
  \[
    \begin{array}{lllclllclll}
                 & &     & \qquad& P(a_1\cap c_1) &=& 0.345& \qquad& P(c_1|a_1) &=& 0.46\\
      P(c_1|b_2) &=& 0.4 &       & P(a_1\cap c_2) &=& 0.405& \qquad& P(c_2|a_1) &=& 0.54\\
      P(c_2|b_2) &=& 0.6 &       & P(a_2\cap c_1) &=& 0.175&       & P(c_1|a_2) &=& 0.7 \\
                 & &     &       & P(a_2\cap c_2) &=& 0.075& \qquad& P(c_2|a_2) &=& 0.3
    \end{array}
  \]
  To simplify calculations, we use the function $H(x) = -x\log_2 x - (1-x)\log_2(1-x)$.
\begin{itemize}
  \item[{\bf a.}]
    This is asking to show that $P(c_i|b_j\cap a_k) = P(c_i|b_j)$ for all $i,j,k = 1,2$.\\
    We could show these 8 equalities explicitly; there is however a short-cut here.\\
    If it is assumed that the student owns a car ($b_1$),
    then that student lives at home, regardless of course performance:
      \[P(c_i|b_1\cap a_k) = P(c_i|b_1)\qquad\text{for all}\:i,k\]
    If the student does not own a car ($b_2$),
    then that student lives at home ($c_1$) with 40\% probability
                        or does not ($c_2$) with 60\% probability,
    regardless of course performance:
      \[P(c_i|b_2\cap a_k) = P(c_i|b_2)\qquad\text{for all}\:i,k\]
  \item[{\bf b.}] Here, we are meant to calculate $I(A,B)$:
    \begin{align*}
      I(A,B) &= H(B) - H(B|A)\\
             &= H(0.2) - (P(a_1)H(B|a_1) + P(a_2)H(B|a_2)\\
             &= H(0.2) - (0.75H(0.1) + 0.25H(0.5))\\
             &\approx 0.120
    \end{align*}
  \item[{\bf c.}] Here, we are meant to calculate $I(A,C)$:
    \begin{align*}
      I(A,C) &= H(C) - H(C|A)\\
             &= H(0.52) - (P(a_1)H(C|a_1) + P(a_2)H(C|a_2)\\
             &= H(0.52) - (0.75H(0.46) + 0.25H(0.7))\\
             &\approx 0.032
    \end{align*}
  \item[{\bf d.}]
  The information in the first digit is $H(A) = H(0.75)\approx 0.811$ bits.\\
  The (new) information in the second digit is $H(B|A) = 0.75H(0.1) + 0.25H(0.5))\approx 0.602$ bits.\\
  The (new or extra) information in the third digit is $H(C|A\cap B)$.\\
  By Part~{\bf a.}, this equals $H(C|B) = 0.2H(1)+0.8H(0.4)\approx0.777$.
\end{itemize}

\bigskip\noindent{\bf 49.}
\begin{itemize}
  \item[{\bf a.}]
    $\ds P(a_j|b_i) = \frac{P(a_j\cap b_i)}{P(b_i)}
                    = \frac{P(b_i|a_j)P(a_j)}{P(b_i)}$\quad
    where
    \begin{align*}
      P(b_1) &= P(b_1|a_1)P(a_1)+P(b_1|a_2)P(a_2)
              = 0.8\times \frac{1}{3} + 0.4\times \frac{2}{3}
              = \frac{1.6}{3}\\
      P(b_2) &= P(b_2|a_1)P(a_1)+P(b_2|a_2)P(a_2)
              = 0.2\times \frac{1}{3} + 0.6\times \frac{2}{3}
              = \frac{1.4}{3}
    \end{align*}
    so
    \begin{align*}
      P(a_1|b_1) &= \frac{P(b_1|a_1)P(a_1)}{P(b_1)}
                  = \frac{0.8\times \frac{1}{3}}{\frac{1.6}{3}}
                  = 0.5\\
      P(a_2|b_1) &= \frac{P(b_1|a_2)P(a_2)}{P(b_1)}
                  = \frac{0.4\times \frac{2}{3}}{\frac{1.6}{3}}
                  = 0.5\\
      P(a_1|b_2) &= \frac{P(b_2|a_1)P(a_1)}{P(b_2)}
                  = \frac{0.2\times \frac{1}{3}}{\frac{1.4}{3}}
                  \approx \frac{1}{7}\\
      P(a_2|b_2) &= \frac{P(b_2|a_2)P(a_2)}{P(b_2)}
                  = \frac{0.6\times \frac{2}{3}}{\frac{1.4}{3}}
                  \approx \frac{6}{7}
    \end{align*}
  \item[{\bf b.}]
    \begin{align*}
      I(A,B) &= H(B) - H(B|A)\\
             &= H(B) - (P(a_1)H(B|a_1) + P(a_2)H(B|a_2)\\
             &= H\Bigl(\frac{1.6}{3}\Bigl) 
              - \Bigl(\frac{1}{3}H(0.8) + \frac{2}{3}H(0.6)\Bigr)\\
             &\approx 0.109
    \end{align*}
\end{itemize}

\bigskip\noindent{\bf 50.}
Let $P(a_1) = x$ and $P(a_2) = 1 - x$.
Then
\begin{align*}
   P(b_1) &= P(b_1|a_1)P(a_1) + P(b_1|a_2)P(a_2)
           = (1-q)x + 0(1-x)
           = (1-q)x\\
   P(b_2) &= P(b_2|a_1)P(a_1)+P(b_2|a_2)P(a_2)
           = 0x + (1-q)(1-x)
           = (1-q)(1-x)\\
   P(b_3) &= P(b_3|a_1)P(a_1)+P(b_3|a_2)P(a_2)
           = qx + q(1-x)
           = q
\end{align*}
    so, writing $p = 1 - q$,
\begin{align*}
   H(B)   &= -P(b_1)\log_2 P(b_1)
             -P(b_2)\log_2 P(b_2)
             -P(b_3)\log_2 P(b_3)\\
          &= -px\log_2 px
             -p(1-x)\log_2 p(1-x)
             -q\log_2 q\\
%          = - px\log_2 p
%            - px\log_2 x
%            - p \log_2 p
%            + px\log_2 p
%            - p \log_2 (1-x)
%            + px\log_2 (1-x)
%            - q\log_2 q
%          = - px\log_2 x
%            - p(1-x)\log_2 (1-x)
%            - p \log_2 p
%            - q \log_2 q
          &= - px\log_2 x
             - p(1-x)\log_2 (1-x)
             - p\log_2 p
             - q\log_2 q\\
          &= pH(x) + H(q)\\[2mm]
   H(B|a_1)&= -P(b_1|a_1)\log_2 P(b_1|a_1)
              -P(b_2|a_1)\log_2 P(b_2|a_1)
              -P(b_3|a_1)\log_2 P(b_3|a_1)\\
           &= -p\log_2 p
              -0\log_2 0
              -q\log_2 q\\
           &= H(q)\\[2mm]
   P(B|a_2)&= -P(b_1|a_2)\log_2 P(b_1|a_2)
              -P(b_2|a_2)\log_2 P(b_2|a_2)
              -P(b_3|a_2)\log_2 P(b_3|a_2)\\
           &= -0\log_2 0
              -p\log_2 p
              -q\log_2 q\\
           &= H(q)\\
   H(B|A)  &=  H(B|a_1)P(a_1)
              +H(B|a_2)P(a_2)\\
           &= H(q)x + H(q)(1-x)\\
           &= H(q)
\end{align*}
Therefore,
\[
  I(A,B) = H(B) - H(B|A)
         = pH(x) + H(q) - H(q)
         = pH(x)
\]
Then $\ds\frac{d}{dx}I(A,B) = p\frac{d}{dx}H(x) = \log_2\Bigl(\frac{1-x}{x}\Bigr)$.\\
Solving $\ds\frac{d}{dx}I(A,B) = 0$, we get $x = \frac{1}{2}$, and so
\[
  C(A,B) = \max_x I(A,B) = pH(\frac{1}{2}) = p = 1 - q
\]

\noindent{\bf 51.}\quad
We can first calculate, regardless of the probabilities $P(a_j)$:
 \begin{align*}
   H(B|a_j)&= \sum_{i=1}^3 (-P(b_i|a_j)\log_2 P(b_i|a_j))
            = -\frac{1}{2}\log_2\frac{1}{2} + 2\Bigl(-\frac{1}{4}\log_2 \frac{1}{4}\Bigr)
            =  \frac{1}{2} + \frac{1}{2}\log_2 4
            = 1.5\\[2mm]
   H(B|A)  &= \sum_{j=1}^3 H(B|a_j)P(a_j)
            =  \frac{3}{2}\times \sum_{j=1}^3 P(a_j)
            = 1.5
\end{align*}
\begin{itemize}
  \item[{\bf a.}]
   \begin{align*}
   H(A)    &= 3\Bigl(-\frac{1}{3}\log_2\frac{1}{3}\Bigr) = \log_2 3\\
   P(b_i)  &= \sum_{j=1}^3 P(b_i|a_j)P(a_j)
            = \frac{1}{2}\times\frac{1}{3} + 2\frac{1}{4}\times\frac{1}{3}
            = \frac{1}{3}
            = P(a_j) \quad\text{for all}\; i,j\\
   H(B)    &= H(A) = \log_2 3\\
  \end{align*}
  Hence,
  \begin{align*}
    I(A,B)&= H(B) - H(B|A) = \log_2 3 - \frac{3}{2} \approx 0.085\\
    H(A|B)&= H(A) - I(A,B) = \frac{3}{2} = 1.5\\
    H(A,B)&= H(A) + H(B|A) = \log_2 3 + \frac{3}{2}\approx 3.085
  \end{align*}
  \item[{\bf b.}]\moveup
   \begin{align*}
    H(A)    &= -\frac{1}{2}\log_2\frac{1}{2}-\frac{1}{3}\log_2\frac{1}{3}-\frac{1}{6}\log_2\frac{1}{6}
             \approx 1.459\\
    P(b_1)  &= \sum_{j=1}^3 P(b_1|a_j)P(a_j)
             = \frac{1}{2}\times\frac{1}{2}
              +\frac{1}{4}\times\frac{1}{3}
              +\frac{1}{4}\times\frac{1}{6}
             = \frac{3}{8}\\
    P(b_2)  &= \sum_{j=1}^3 P(b_2|a_j)P(a_j)
             = \frac{1}{4}\times\frac{1}{2}
              +\frac{1}{2}\times\frac{1}{3}
              +\frac{1}{4}\times\frac{1}{6}
             = \frac{1}{3}\\
    P(b_3)  &= \sum_{j=1}^3 P(b_3|a_j)P(a_j)
             = \frac{1}{4}\times\frac{1}{2}
              +\frac{1}{4}\times\frac{1}{3}
              +\frac{1}{2}\times\frac{1}{6}
             = \frac{7}{24}\\
    H(B)    &= -\frac{3}{8}\log_2 \frac{3}{8} -\frac{1}{3}\log_2 \frac{1}{3} -\frac{7}{24}\log_2 \frac{7}{24}
             \approx 1.577
  \end{align*}
  Hence,
  \begin{align*}
    I(A,B)&= H(B) - H(B|A) \approx 1.577 - 1.5   = 0.077\\
    H(A|B)&= H(A) - I(A,B) \approx 1.459 - 0.077 = 1.382\\
    H(A,B)&= H(A) + H(B|A) \approx 1.459 + 1.5   = 2.959
  \end{align*}
  \item[{\bf c.}] Your guess is your guess :)\\
  \item[{\bf d.}] Write $P(a_1) = x_1$ and $P(a_2) = x_2$.
   \begin{align*}
    P(b_1)  &= \sum_{j=1}^3 P(b_1|a_j)P(a_j)
             = \frac{1}{2}x_1
              +\frac{1}{4}x_2
              +\frac{1}{4}(1-x_1-x_2)
             = \frac{1}{4}(1+x_1)\\
    P(b_2)  &= \sum_{j=1}^3 P(b_2|a_j)P(a_j)
             = \frac{1}{4}x_1
              +\frac{1}{2}x_2
              +\frac{1}{4}(1-x_1-x_2)
             = \frac{1}{4}(1+x_2)\\
    P(b_3)  &= \sum_{j=1}^3 P(b_3|a_j)P(a_j)
             = \frac{1}{4}x_1
              +\frac{1}{4}x_2
              +\frac{1}{2}(1-x_1-x_2)
             = \frac{1}{4}(2-x_1-x_2)
   \end{align*}
   \begin{align*}
    H(B)    &=-\frac{1}{4}(1+x_1)    \log_2 \frac{1}{4}(1+x_1)
              -\frac{1}{4}(1+x_2)    \log_2 \frac{1}{4}(1+x_2)
              -\frac{1}{4}(2-x_1-x_2)\log_2 \frac{1}{4}(2-x_1-x_2)\\
%%            &=-\frac{1}{4}           \log_2 \frac{1}{4}(1+x_1)
%%              -\frac{1}{4}x_1        \log_2 \frac{1}{4}(1+x_1)
%%              -\frac{1}{4}           \log_2 \frac{1}{4}(1+x_2)
%%              -\frac{1}{4}x_2        \log_2 \frac{1}{4}(1+x_2)
%%              -\frac{1}{2}           \log_2 \frac{1}{4}(2-x_1-x_2)
%%              +\frac{1}{4}x_1        \log_2 \frac{1}{4}(2-x_1-x_2)
%%              +\frac{1}{4}x_2        \log_2 \frac{1}{4}(2-x_1-x_2)\\
%            &=-\frac{1}{4}           \log_2 \frac{1}{4}
%              -\frac{1}{4}           \log_2 (1+x_1)
%              -\frac{1}{4}x_1        \log_2 \frac{1}{4}
%              -\frac{1}{4}x_1        \log_2 (1+x_1)
%              -\frac{1}{4}           \log_2 \frac{1}{4}\\
%            &\;\;\; -\frac{1}{4}           \log_2 (1+x_2)
%              -\frac{1}{4}x_2        \log_2 \frac{1}{4}
%              -\frac{1}{4}x_2        \log_2 (1+x_2)
%              -\frac{1}{4}           \log_2 \frac{1}{4}\\
%            &\;\;\; -\frac{1}{2}           \log_2 (2-x_1-x_2)
%              +\frac{1}{4}x_1        \log_2 \frac{1}{4}
%              +\frac{1}{4}x_1        \log_2 (2-x_1-x_2)
%              +\frac{1}{4}x_2        \log_2 \frac{1}{4}
%              +\frac{1}{4}x_2        \log_2 (2-x_1-x_2)\\
%            &= 2
%              -\frac{1}{4}(1+x_1)    \log_2 (1+x_1)
%              -\frac{1}{4}(1+x_2)    \log_2 (1+x_2)
%              -\frac{1}{4}(2-x_1-x_2)\log_2 (2-x_1-x_2)\\
%            &= 2 + \frac{1}{4}\bigl(H(1+x_1) + H(1+x_2) + H(2-x_1-x_2)\bigr)
             &= I\Bigl(\frac{1+x_1}{4}\Bigr) + I\Bigl(\frac{1+x_2}{4}\Bigr) + I\Bigl(\frac{2-x_1-x_2}{4}\Bigr)
   \end{align*}
  where $I(x) = -x\log_2 x$.
  Hence,
  \[
    I(A,B) = H(B) - H(B|A) = I\Bigl(\frac{1+x_1}{4}\Bigr) + I\Bigl(\frac{1+x_2}{4}\Bigr) + I\Bigl(\frac{2-x_1-x_2}{4}\Bigr) - \frac{3}{2}
  \]
\end{itemize}
To find the maximum of this function,
we solve the equations $\ds\frac{d}{dx_1}I(A,B) = 0$ and $\ds\frac{d}{dx_y}I(A,B) = 0$.
Let us first solve the first equation:
\begin{align*}
  0 &= \frac{d}{dx_1}I(A,B)
     = \frac{1}{4}\log_2\Bigl(\frac{2-x_1-x_2}{1+x_1}\Bigr)\\[3mm]
  \text{so}\quad
  2 - x_1 - x_2
    &= 1 + x_1
\end{align*}
and so $x_1 =\frac{1}{2}(1-x_2)$.
Since $I(A,B)$ is symmetric with respect to $x_1$ and $x_2$,
the second equation will imply that $x_2 = \frac{1}{2}(1-x_1)$,
so $x_1 = \frac{1}{2}(1-\frac{1}{2}(1-x_1)) = \frac{1}{4} + \frac{1}{4}x_1$ and so $x_1 = \frac{1}{3}$.
Hence, $x_2 = 1-x_1-x_2 = \frac{1}{3}$,
so we find that $I(A,B)$ is maximal for the probabilities $P(a_j) = \frac{1}{3}$ from part {\bf a}.,
and that the capacity is
\[
  C(A,B) = \max_{x_1,x_2} I(A,B) = H(B) - H(B|A) = \log_2 3 - \frac{3}{2} \approx 0.085
\]

\noindent{\bf 52.}\quad
\[
  P(a_j|b_i) = \frac{P(a_j\cap b_i)}{P(b_i)}
             = \frac{P(b_i|a_j)P(a_j)}{P(b_i)}
\]
where
\begin{align*}
  P(b_1) &= P(b_1|a_1)P(a_1) + P(b_1|a_2)P(a_2) = \frac{5}{7}x\\
  P(b_2) &= P(b_2|a_1)P(a_1) + P(b_2|a_2)P(a_2) = \frac{2}{7}x\\
  P(b_3) &= P(b_3|a_1)P(a_1) + P(b_3|a_2)P(a_2) = \frac{1}{10}(1-x)\\
  P(b_4) &= P(b_4|a_1)P(a_1) + P(b_4|a_2)P(a_2) = \frac{9}{10}(1-x)
\end{align*}
so
\begin{align*}
  P(a_1|b_1) &= \frac{P(b_1|a_1)P(a_1)}{P(b_1)} = 1\\
  P(a_1|b_2) &= \frac{P(b_2|a_1)P(a_1)}{P(b_2)} = 1\\
  P(a_1|b_3) &= \frac{P(b_3|a_1)P(a_1)}{P(b_3)} = 0\\
  P(a_1|b_4) &= \frac{P(b_4|a_1)P(a_1)}{P(b_4)} = 0\\
  P(a_2|b_1) &= \frac{P(b_1|a_2)P(a_2)}{P(b_1)} = 0\\
  P(a_2|b_2) &= \frac{P(b_2|a_2)P(a_2)}{P(b_2)} = 0\\
  P(a_2|b_3) &= \frac{P(b_3|a_2)P(a_2)}{P(b_3)} = 1\\
  P(a_2|b_4) &= \frac{P(b_4|a_2)P(a_2)}{P(b_4)} = 1
\end{align*}
\begin{itemize}
  \item[{\bf a.}] By the above, $H(a_j|b_i) = 0$ for all $i,j$, so $H(A|B) = 0$.\\
    This reflects that the elements of $B$ imply which elements of $A$ are given;
    in other words, there is no information in $A$ that is not determined by $B$.\\
    In particular, if $b_1$ or $b_2$ are received, then $a_1$ has been sent;
    otherwise if $b_3$ or $b_4$ are received, then $a_2$ has been sent.
  \item[{\bf b.}] $I(A,B) = H(A) - H(A|B) = H(A) = H(x)$, which has maximum
    \[
      C(A,B) = \max_{x} I(A,B) = H(\frac{1}{2}) = 1
    \]
\end{itemize}

\bigskip\noindent{\bf 53.}
The block code $\mathbb{Z}_2^{15}$ contains $2^{15} = 32768$
which is enough to encode the students,
for instance by a binary decision tree.

\bigskip\noindent{\bf 54.}
\begin{itemize}
  \item[a)]
    First use the Euclidean Algorithm forwards:
    \begin{align*}
      3876 &= 11\times 324 + 312\\
       324 &=  1\times 312 +  12\\
       312 &= 26\times  12 +   0\,,
    \end{align*}
    so $d = \gcd(312,3876) = 12$.
    Now use the Euclidean Algorithm backwards:
    \begin{align*}
      12 &= 324 -   312\\
         &= 324 - (3876 - 11\times324)\\
         &= 12\times 324 - 3876
    \end{align*}
    Hence, $12 = \gcd(324,3876) = 324x + 3876$ for $x = 12$ and $y = -1$.
  \item[b)]
    First use the Euclidean algorithm forwards:
    \begin{align*}
      7412 &= 4\times 1513 + 1360\\
      1513 &= 1\times 1360 +  153\\
      1360 &= 8\times  153 +  136\\
       153 &= 1\times  136 +   17\\
       136 &= 8\times   17 +  0\,,
    \end{align*}
    so $d = \gcd(7412,1513) = 17$.
    Now use the Euclidean algorithm backwards:
    \begin{align*}
       17 &= 153 - 136\\
          &= 153 - (1360 - 8\times 153)\\
          &=  9\times   153 - 1360\\
          &=  9\times (1513 - 1360) - 1360\\
          &=  9\times  1513 - 10\times1360\\
          &=  9\times  1513 - 10\times(7412 - 4\times 1513)\\
          &= 49\times  1513 - 10\times 7412
    \end{align*}
    Hence, $d = \gcd(7412,1513) = 7412x + 1513y$ for $x = -10$ and $y = 49$.
  \item[c)]
    First use the Euclidean algorithm forwards:
    \begin{align*}
      2187 &= 2\times 1024 + 139\\
      1024 &= 7\times  139 +  51\\
       139 &= 2\times   51 +  37\\
        51 &= 1\times   37 +  14\\
        37 &= 2\times   14 +   9\\
        14 &= 1\times    9 +   5\\
         9 &= 1\times    5 +   4\\
         5 &= 1\times    4 +   1
    \end{align*}
    so $\gcd(1024,2187) = 1$.
    Now use the Euclidean algorithm backwards:
    \begin{align*}
       1 &=   5 - 4\\
         &=   5 - (9 - 5)\\
         &=   2\times 5 - 9\\
         &=   2\times(14 - 9) - 9\\
         &=   2\times 14 - 3\times 9\\
         &=   2\times 14 - 3\times(37 - 2\times 14)\\
         &=   8\times 14 - 3\times 37\\
         &=   8\times(51-37) - 3\times 37\\
         &=   8\times 51 - 11\times 37\\
         &=   8\times 51 - 11\times(139 - 2\times 51)\\
         &=  30\times 51 - 11\times 139\\
         &=  30\times(1024 - 7\times 139) - 11\times 139\\
         &=  30\times 1024 - 221\times 139\\
         &=  30\times 1024 - 221\times(2187-2\times 1024)\\
         &=  472\times 1024 - 221\times 2187
    \end{align*}
    Hence, $1 = \gcd(1024,2187) = 1024x + 2187y$ for $x = 472$ and $y = -221$.
\end{itemize}

\bigskip\noindent{\bf 55.}
\[
  \begin{array}{rl}
    \mathbb{U}_{24} &= \{ 1,5,7,11,13,17,19,23\}\\
    \mathbb{U}_{36} &= \{ 1,5,7,11,13,17,19,23,25,29,31,35\}\\
    \mathbb{U}_{17} &= \{ 1,\ldots 16\}
  \end{array}
  \qquad\textrm{so}\qquad
  \begin{array}{rl}
    \varphi(24) &= |\mathbb{U}_{24}| = 8\\
    \varphi(36) &= |\mathbb{U}_{24}| = 12\\
    \varphi(17) &= |\mathbb{U}_{17}| = 16
  \end{array}
\]

\noindent{\bf 56.}
\[
  \begin{array}{rl}
        72 &= 2^3\cdot3^2\\
      1224 &= 2^3\cdot3^2\cdot 17 = 72\cdot 17\\
    561561 &= 3  \cdot 7 \cdot 11^2 \cdot 13 \cdot 17\\
  \end{array}
  \qquad\text{so}\qquad
  \begin{array}{rl}
    \phi(72)     &= \varphi(2^3)\varphi(3^2) = (2^3 - 2^2)(3^2 - 3^1) = 4\times 6 = 24\\
    \phi(1224)   &= \varphi(72)\varphi(17) = 24\times 16 = 384\\
    \phi(561561) &= \varphi(3)\varphi(7)\varphi(11^2)\varphi(13)\varphi(17)\\
                 &= 2\cdot 6\cdot (11^2-11^1)\cdot12\cdot16
                  = 253440
  \end{array}
\]

\noindent{\bf 57.}
\[
  \begin{array}{|c|ccccc|}\hline
    + & 0 & 1 & 2 & 3 & 4\\\hline
    0 & 0 & 1 & 2 & 3 & 4\\
    1 & 1 & 2 & 3 & 4 & 0\\
    2 & 2 & 3 & 4 & 0 & 1\\
    3 & 3 & 4 & 0 & 1 & 2\\
    4 & 4 & 0 & 1 & 2 & 3\\\hline
  \end{array}
  \qquad
  \begin{array}{|c|ccccc|}\hline
    \times
      & 0 & 1 & 2 & 3 & 4\\\hline
    0 & 0 & 0 & 0 & 0 & 0\\
    1 & 0 & 1 & 2 & 3 & 4\\
    2 & 0 & 2 & 4 & 1 & 3\\
    3 & 0 & 3 & 1 & 4 & 2\\
    4 & 0 & 4 & 3 & 2 & 1\\\hline
  \end{array}
  \qquad
  \begin{array}{|c|cccccc|}\hline
    + & 0 & 1 & 2 & 3 & 4 & 5\\\hline
    0 & 0 & 1 & 2 & 3 & 4 & 5\\
    1 & 1 & 2 & 3 & 4 & 5 & 0\\
    2 & 2 & 3 & 4 & 5 & 0 & 1\\
    3 & 3 & 4 & 5 & 0 & 1 & 2\\
    4 & 4 & 5 & 0 & 1 & 2 & 3\\
    5 & 5 & 0 & 1 & 2 & 3 & 4\\\hline
  \end{array}
  \qquad
  \begin{array}{|c|cccccc|}\hline
    \times
      & 0 & 1 & 2 & 3 & 4 & 5\\\hline
    0 & 0 & 0 & 0 & 0 & 0 & 0\\
    1 & 0 & 1 & 2 & 3 & 4 & 5\\
    2 & 0 & 2 & 4 & 0 & 2 & 4\\
    3 & 0 & 3 & 0 & 3 & 0 & 3\\
    4 & 0 & 4 & 2 & 0 & 3 & 2\\
    5 & 0 & 5 & 4 & 3 & 2 & 1\\\hline
  \end{array}
\]
Every non-zero element in $\mathbb{Z}_5$ has an inverse and is therefore a unit,
\\in contrast to the elements 3, 4, and 5 in $\mathbb{Z}_6$.
\\Therefore,  $\mathbb{Z}_5$ is a field and  $\mathbb{Z}_6$ is not.


\bigskip\noindent{\bf 58.}
\begin{itemize}
  \item[a)]\moveup
    \begin{align*}
         6x&\equiv  7\mymod{17}\\\Leftrightarrow\quad
         6x&\equiv 24\mymod{17}\\\Leftrightarrow\quad\ph{6}
          x&\equiv  4\mymod{17}\quad\textrm{(since $\gcd(6,17)=1$)}
    \end{align*}
  \item[b)]\moveup
    \begin{align*}
         6x&\equiv  8\mymod{11}\\\Leftrightarrow\quad
         6x&\equiv 30\mymod{11}\\\Leftrightarrow\quad\ph{6}
          x&\equiv  5\mymod{11}\quad\textrm{(since $\gcd(6,11)=1$)}
    \end{align*}
  \item[c)]\moveup
    \begin{align*}
         6x&\equiv  9\mymod{13}\\\Leftrightarrow\quad
         2x&\equiv  3\mymod{13}\quad\textrm{(since $\gcd(3,13)=1$)}\\\Leftrightarrow\quad
         2x&\equiv 16\mymod{13}\\\Leftrightarrow\quad\ph{2}
          x&\equiv  8\mymod{13}\quad\textrm{(since $\gcd(2,13)=1$)}
    \end{align*}
\end{itemize}


\bigskip\noindent{\bf 59.}
\begin{itemize}
  \item[a)] In $\mathbb{Z}_{11}$, $6\times 2 = 12 = 1$ , so $6^{-1} = 2$.
  \item[b)] $\gcd(6,10) = 2\neq 1$, so 6 has no inverse in $\mathbb{Z}_{10}$.
  \item[c)] In $\mathbb{Z}_{23}$, $6\times 4 = 24 = 1$ , so $6^{-1} = 4$.
\end{itemize}




\end{document}



















