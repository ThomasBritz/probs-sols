\documentclass[11pt]{article}
\usepackage{amsmath,amssymb,latexsym}

\parindent 0in
\addtolength{\textwidth}{52mm}
\addtolength{\oddsidemargin}{-28mm}
\addtolength{\evensidemargin}{-26mm}
\addtolength{\topmargin}{-20mm}
\addtolength{\textheight}{40mm}

\newcommand{\ds}{\displaystyle}
\renewcommand{\vec}[1]{\mathbf{#1}}
\newcommand{\vc}[1]{\begin{pmatrix}#1\end{pmatrix}}
\newcommand{\tmat}[1]{{\left(\begin{array}{rrr}#1\end{array}\right)}}
\newcommand{\qmat}[1]{{\left(\begin{array}{rrrr}#1\end{array}\right)}}
\newcommand{\pmat}[1]{{\left(\begin{array}{rrrrr}#1\end{array}\right)}}
\newcommand{\proof}{{\sc Proof.}\quad}
\newcommand{\qed}{\quad$\square$}

\newcommand{\moveup}{\begin{picture}(0,0)(0,0)\end{picture}\vspace*{-8.15mm}}
\newcommand{\upabit}{\begin{picture}(0,0)(0,0)\end{picture}\vspace*{-5mm}}


\date{}
\author{}
\title{\sc Solutions to MATH3411 Problems 1--9}


\begin{document} \maketitle

\vspace*{-10mm}

\noindent{\bf 1.}
\begin{itemize}
  \item[{a)}]
     Since $n$ is not a prime,
     we can write $n = km$ where $k,m>1$ are integers.
     Suppose that $k,m>\sqrt{n}$;
     then $n = km > (\sqrt{n})^2 = n$,
     a contradiction.
     We see that either $k\leq\sqrt{n}$ or $m\leq\sqrt{n}$;
     in particular, one of these integers - and thus $n$ -
     then has a proper factor less than or equal to $\sqrt{n}$.
  \item[{b)}]
     Assume that $n$ is not a prime.
     By part a), $n$ has a prime factor less than or equal to 20.
     The prime factors of $51051$ are 3, 7, 11, 13, and 17,
     so if $\text{gcd}\,(n,51051) = 1$,
     then none of these prime factors divide $n$.
     If $n$ furthermore does not have 2 or 5 as a prime factor,
     then this just leaves 19 as $n$'s only prime factor,
     so $n$ is a power of 19, namely 19, $19^2 = 361$, $19^3$, or greater.
     However, $18\leq n\leq 400$ and $n\geq 361$,
     so none of these are possible.

     We conclude that $n$ must be a prime.
  \item[{c)}]
     We just have to check whether any of the numbers 2, 3, 5, 7, 11, 13, 17
     divide $n=323$ and $n=373$, respectively.
     We find that none divide 373 (so it is prime) but that $323 = 17\times 19$.
\end{itemize}

\noindent{\bf 2.} (Extended result, proving that if $a^n-1$ is prime, then $a=2$ and $n$ is prime.)\\
Suppose that $a^n-1$ is prime and write $n = pq$ where $p$~is a prime factor of~$n$.
Now,
\[
  \begin{array}{rccrrrrrrrrrr}
    a^{pq} - 1 &=& a^{pq}&+&a^{(p-1)q}&+&\cdots&+&a^{q}& &\\
               & &       &-&a^{(p-1)q}&-&\cdots&-&a^{q}&-&1&=&(a^q\!-\!1)(a^{(p-1)q}+\cdots+a^{q}+1)
  \end{array}
\]

Since $a^n-1=a^{pq}-1$ is prime,
one of the two terms above must be~1.
Since $a^{(p-1)q}+\cdots+a^{q}+1>1$,
we have $a^q-1=1$.
Therefore,
$a^q=2$,
so $a=2$ and $q=1$,
and $n=p$ is prime.

Suppose that $2^n+1$ is prime and write $n = rt$ where $r,t>1$ are positive integers.\\
Assume that $n$ is not a power of 2; then $r$ or $s$ is odd, say $s$.
%Now, for any positive integers $a$, $b$, and $m$,
%\[
%  \begin{array}{rccrrrrrrrrrr}
%    a^m - b^m &=& a^m&+&a^{m-1}b&+&\cdots&+&ab^{m-1}\\
%              & &    &-&a^{m-1}b&-&\cdots&-&ab^{m-1}& -&b^n
%              &=&(a-b)(a^{m-1} + a^{m-1}b + \cdots + ab^{m-2} + b^{m-1})\end{array}
%\]
%Therefore, $a - b \mid a^m - b^m$.
%Set $a = 2^r$, $b = -1$, and $m=s$;
Now, for any positive integer $a$, 
\[
  \begin{array}{rccrrrrrrrrrrrrrr}
    a^s + 1  &=& a^s&-&a^{s-1}&+&a^{s-1}&\cdots&-&a^2&+&a\\
             & &    &+&a^{s-1}&-&a^{s-1}&\cdots&+&a^2&-&a&+&1
             &=&(a+1)(a^{s-1} - a^{s-2} + \cdots - a + 1)\end{array}
\]
Therefore, $a + 1 \mid a^s + 1$.
Set $a = 2^r$;
we then see that $2^r + 1$ divides $(2^r)^s + 1 = 2^n + 1$.
But $r<n$, so $2^r+1$ cannot equal $2^n+1$ and is therefore a factor of $2^n+1$,
which means that $2^n + 1$ is not prime, a contradiction.

We conclude that $n$ is a power of~2.

\bigskip
\noindent{\bf 3.} \moveup
\begin{itemize}
  \item[{a)}] $\frac{1}{13}$
  \item[{b)}] $\frac{1}{3}$
  \item[{c)}] $\frac{1}{13}$
  \item[{d)}] ``Pick a Queen" and ``pick a face card" are both independent of ``pick a black card"
              but are not independent of each other.
\end{itemize}

\bigskip
\noindent{\bf 4.}
(It is perhaps easiest for understanding to draw a tree diagram here but I'll be lazy.)
\[
         P(\text{0 received}|\text{0 sent})
   = 1 - P(\text{1 received}|\text{0 sent})
   = 1 - 0.1
   = 0.9
\]
so
\begin{align*}
      P(0\,\text{received})
  &=  P(\text{0 received and 0 sent})
    + P(\text{0 received and 1 sent})\\
  &=  P(\text{0 received}|\text{0 sent})P(\text{0 sent})
    + P(\text{0 received}|\text{1 sent})P(\text{1 sent})\\
  &=  0.9\times 0.5
    + 0.2\times 0.5 \qquad\text{(since $P(\text{0 sent}) = P(\text{1 sent}) = \textstyle\frac{1}{2}$)}\\
  &= 0.55
\end{align*}
Now, $P(\text{0 received and 0 sent}) = P(\text{0 received}|\text{0 sent})P(\text{0 sent})$,
so
\[
    P(\text{0 received}|\text{0 sent})
  = \frac{P(\text{0 received and 0 sent})}{P(\text{0 sent})}
  = \frac{P(\text{0 received}|\text{0 sent})P(\text{0 sent})}{P(\text{0 sent})}
  = \frac{0.9\times 0.5}{0.55}
  \approx 0.82
\]


\noindent{\bf 5.}
Call the door the contestant chooses Door~1
and the door opened by host Door~2.
Let $W_i$ be the event ``major prize behind door $i$'',
and $H_i$ be ``host opens door $i$''.
We know that $P(W_i)=\frac{1}{3}$ for each $i$ and that
\[
    P(H_2\mid W_2)
  = 0\qquad P(H_2\mid W_3)
  = 1\qquad P(H_2\mid W_1)
  = 0.5\,.
\]
We want to find $P(W_3 \mid H_2)$.
Now
\[
    P(H_2)
  = P(H_2\mid W_1) P(W_1) + P(H_2\mid W_2)P(W_2) + P(H_2\mid W_3)P(W_3)
  = \frac12\times\frac13 + 0\times\frac13 + 1\times\frac13
  = \frac12
\]
so (by Bayes' rule)
\[
    P(W_3 \mid H_2)
  = \frac{P(H_2 \mid W_3)P(W_3)}{P(H_2)}
  = \frac{1/3}{1/2}
  = \frac23
\]
Given that
\[
  P(W_1 \mid H_2) = 1 - P(W_2 \mid H_2) - P(W_3 \mid H_2) = 1 - 0 - \frac{2}{3} = \frac{1}{3}\,,
\]
we see that the prize is twice as likely to be behind the third door: so swap.

\newpage
\noindent{\bf 6.}\moveup
\begin{itemize}
  \item[{a)}] $\ds\binom{n}{k}p^k(1-p)^{n-k}$
  \item[{b)}] $\ds\sum_{j=0}^{\lfloor\frac{n}{2}\rfloor}\binom{n}{2j}p^{2j}(1-p)^{n-2j}$
  \item[{c)}] Setting $q = 1 - p$ and using the Binomial Theorem,
  \begin{align*}
       \frac{  1     + (1-2p)^n}{2}
    &= \frac{(q+p)^n + (q-p)^n}{2}\\
    &= \frac{1}{2}\Bigl(
         \sum_{k=0}^n \binom{n}{k}p^k q^{n-k}
        +\sum_{k=0}^n \binom{n}{k}(-p)^k q^{n-k}\Bigr)\\
    &= \frac{1}{2}\Bigl(
         \sum_{j=0}^{\lfloor\frac{n}{2}\rfloor} \binom{n}{2j}(p^{2j}+p^{2j}) q^{n-2j}
        +\sum_{j=0}^{\lfloor\frac{n-1}{2}\rfloor}  \binom{n}{2j+1}(p^{2j+1}-p^{2j+1}) q^{n-{2j+1}}\Bigr)\\
    &= \frac{1}{2}
         \sum_{j=0}^{\lfloor\frac{n}{2}\rfloor} \binom{n}{2j} 2 p^{2j} q^{n-2j} + 0\\
    &=   \sum_{j=0}^{\lfloor\frac{n}{2}\rfloor} \binom{n}{2j} p^{2j} q^{n-2j}\\
    &=   \sum_{j=0}^{\lfloor\frac{n}{2}\rfloor} \binom{n}{2j} p^{2j}(1-p)^{n-2j}
  \end{align*}
    We recognise this as the sum in part b).
\end{itemize}

\bigskip
\noindent{\bf 7.}\moveup
\begin{itemize}
  \item[{a)}] $(1-p)^n = .999^{100} \approx 0.9048$
  \item[{b)}] By parts b) and c) in Problem {\bf 6},
    we see that the probability of an undetected error
    (i.e., the probability of an even, non-zero number of errors)
    is
    \[
        \frac{1 + (1-2p)^n}{2} - (1-p)^n
      = 0.9092834024 - 0.9047921471
      = 0.0044912553
      \approx 0.0045
    \]
  \end{itemize}

\bigskip
\noindent{\bf 8.}\moveup
  \[
     (1\times0
    + 2\times5
    + 3\times5
    + 4\times2
    + 5\times0
    + 6\times8
    + 7\times6
    + 8\times3
    + 9\times8)
    \!\!\mod 11
    = (219 \!\!\mod 11)
    = 10
  \]
  so the first number is a valid ISBN.
  In contrast,
  \[
     (1\times0
    + 2\times5
    + 3\times7
    + 4\times6
    + 5\times0
    + 6\times8
    + 7\times3
    + 8\times1
    + 9\times4)
    \!\!\mod 11
    = (168 \!\!\mod 11)
    = 0
  \]
  so the second number is not an ISBN;
  the correct check digit would have been 0.


\newpage
\noindent{\bf 9.}\moveup
\begin{itemize}
  \item[{a)}] Since $\ds
       0
     \equiv \sum^{10}_{i=1} i x_i
     \equiv \sum^{9}_{i=1} i x_i + 10 x_{10}
     \equiv \sum^{9}_{i=1} i x_i - x_{10} \pmod{11}$,
    we see that $x_{10}$ is given by $x_1,\ldots,x_9$:
    \[
      x_{10} = \sum^{9}_{i=1} i x_i \mod{11}
    \]
    (If $x_{10} = 10$, then the word is not a valid codeword, so we don't count it.)\\
    Adding the two congruences gives the congruence
    \[
      \sum^{10}_{i=1} (i+1) x_i \equiv 0 \pmod{11}\,\qquad
      \text{or, since } 11\equiv 0\pmod{11}\,,\qquad
      \sum^{9}_{i=1} (i+1) x_i \equiv 0 \pmod{11}\,.
    \]
    As with $x_{10}$, we see that $2x_1$ and thus $x_1$ depends on $x_2,\ldots,x_9$;
    in particular,
    \[
      2x_1 \equiv -\sum^{9}_{i=2} (i+1) x_i \pmod{11}
    \]
    so, since $2^{-1} = 6$ and $-6 = 5$ in $\mathbb{Z}_{11}$,
    \[
      x_1 \equiv 5\sum^{9}_{i=2} (i+1) x_i \mod{11}\,.
    \]
    We can choose the 8 digits $x_2,\ldots,x_9$ (almost) freely,
    apart from the estimated $\frac{1}{11}\approx 0.9$ of the time when the congruences will not be satisfied,
    but then $x_1$ and $x_{10}$ are fixed.
    Therefore, a rough estimation for $|\mathcal{C}|$ is $10^8$.
    A slightly better one might be $9\times 10^7$.
  \item[{b)}]
    Suppose that ${\bf x} = x_1 \cdots x_{10}\in\mathcal{C}$ is sent
    and     that ${\bf y} = y_1 \cdots y_{10}$ is received.\\
    Now assume that exactly 1 error has occured,
    changing $x_k$ to $y_k = x_k + m$ for some $k$ and $m$.\\
    Then since ${\bf x}\in\mathcal{C}$,
    $\ds
      0\equiv \sum^{10}_{i=1} x_i
       \equiv \sum^{10}_{i=1} y_i - m\pmod{11}$,
    we see that $\ds m = \sum^{10}_{i=1} y_i \mod{11}$.\\
    Similarly,
    $\ds
      0\equiv \sum^{10}_{i=1} i x_i
       \equiv \sum^{10}_{i=1} i y_i - km\pmod{11}$,
    so $\ds km \equiv \sum^{10}_{i=1} i y_i\pmod{11}$.\\
    Since 11 is prime,
    we can thus determine $\ds k = m^{-1}\sum^{10}_{i=1} i y_i\mod{11}$.\\
    We now know which digit ($k$) is incorrect
    and by how much it is incorrect ($m$),
    so we can correct it.

    To show that the code can also detect the error caused by a swapping two
    digits, just re-use the course notes/slides proof of this property
    for ISBN.
  \item[{c)}] Use part b): $\ds
      m = \sum^{10}_{i=1} y_i \mod{11}
        = (0+6+8+0+2+7+1+3+8+5) \mod{11} = 7$.\\
    The inverse of 7 in $\mathbb{Z}_{11}$ is 8,
    so
    \begin{align*}
      k&= m^{-1}\sum^{10}_{i=1} i y_i\mod{11}
        = 8\sum^{10}_{i=1} i y_i\mod{11}\\
       &= 8(1\times 0
          + 2\times 6
          + 3\times 8
          + 4\times 0
          + 5\times 2
          + 6\times 7
          + 7\times 1
          + 8\times 3
          + 9\times 8
          +10\times 5)\mod{11}\\
       &= 3
    \end{align*}
    In other words, the 3rd digit is wrong
    and is 7 too big, modulo 11: it should be $8 - 7 = 1$ (in $\mathbb{Z}_{11}$).
    The corrected number is then $0610271385$.
\end{itemize}

\end{document}
