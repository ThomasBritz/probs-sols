\documentclass[11pt]{article}
\usepackage{amsmath,amssymb,latexsym}
%\usepackage{amsmath,amssymb,latexsym,pstricks}

\parindent 0in
\addtolength{\textwidth}{52mm}
\addtolength{\oddsidemargin}{-28mm}
\addtolength{\evensidemargin}{-26mm}
\addtolength{\topmargin}{-20mm}
\addtolength{\textheight}{40mm}

\newcommand{\ph}{\phantom}
\newcommand{\ds}{\displaystyle}
\newcommand{\tvs}{\textvisiblespace}
\newcommand{\mymod}[1]{\,(\textrm{mod}\,#1)}
\renewcommand{\vec}[1]{\mathbf{#1}}
\newcommand{\vc}[1]{\begin{pmatrix}#1\end{pmatrix}}
\newcommand{\tmat}[1]{{\left(\begin{array}{rrr}#1\end{array}\right)}}
\newcommand{\qmat}[1]{{\left(\begin{array}{rrrr}#1\end{array}\right)}}
\newcommand{\pmat}[1]{{\left(\begin{array}{rrrrr}#1\end{array}\right)}}
\newcommand{\proof}{{\sc Proof.}\quad}
\newcommand{\qed}{\quad$\square$}

%\newcommand{\myup}{\begin{picture}(0,0)(0,20)\ph{bla}\end{picture}}
\newcommand{\moveup}{\begin{picture}(0,0)(0,0)\end{picture}\vspace*{-8.15mm}}
\newcommand{\upabit}{\begin{picture}(0,0)(0,0)\end{picture}\vspace*{-5mm}}

\newcommand{\ra}[1]{\xrightarrow{#1}}
\newcommand{\dra}[2]{\begin{array}{c}\xrightarrow{\text{\scriptsize$#1$}}\\[-2mm]{\text{\scriptsize$#2$}}\end{array}}


\date{}
\author{}
\title{\sc Solutions to MATH3411 Problems 76--83}


\begin{document} \maketitle

\vspace*{-10mm}

\noindent{\bf 76.}
We first factor $n = 8051$:
\[
  \begin{array}{llc}
    x_0 \equiv 1         &                       & \gcd(x_{2i} - x_i, n)\\
    x_1 \equiv 2\\
    x_2 \equiv 5         & x_2 - x_1 \equiv 3    &                  1\\
    x_3 \equiv 26\\
    x_4 \equiv 677       & x_4 - x_2 \equiv 672  &                  1\\
    x_5 \equiv 7474\\
    x_6 \equiv 2839      & x_6 - x_3 \equiv 2813 &                 97
  \end{array}
\]
We have found a factor $d = 97$.
It is a prime and so is $\dfrac{n}{d} = 83$,
so $n = 91643 = 83\times 97$.

Now let us factor $n = 201001$:
\[
  \begin{array}{llc}
    x_0 \equiv 1         &                       & \gcd(x_{2i} - x_i, n)\\
    x_1 \equiv 2\\
    x_2 \equiv 5         & x_2 - x_1 \equiv 3    &                  1\\
    x_3 \equiv 26\\
    x_4 \equiv 677       & x_4 - x_2 \equiv 672  &                  1\\
    x_5 \equiv 56328\\
    x_6 \equiv 42800     & x_6 - x_3 \equiv 42774&                  1\\
    x_7 \equiv 117888\\
    x_8 \equiv 170404    & x_8 - x_4 \equiv 169727&                19
  \end{array}
\]
We have found a prime factor $d = 19$ of $n = 201001$.\\
We now factor the quotient $\dfrac{n}{d} = 10579$:
\[
  \begin{array}{llc}
    x_0   \equiv 1    &                            & \gcd(x_{2i} - x_i, n)\\
    x_1   \equiv 2\\
    x_2   \equiv 5    & x_2    - x_1   \equiv 3    &                  1\\
    x_3   \equiv 26\\
    x_4   \equiv 677  & x_4    - x_2   \equiv 672  &                  1\\
    x_5   \equiv 3433 \\
    x_6   \equiv 484  & x_6    - x_3   \equiv 458  &                  1\\
    x_7   \equiv 1519 \\
    x_8   \equiv 1140 & x_8    - x_4   \equiv 463  &                  1\\
    x_9   \equiv 8963 \\
    x_{10}\equiv 9023 & x_{10} - x_5   \equiv 5590 &                  1\\
    x_{11}\equiv 9125 \\
    x_{12}\equiv 8896 & x_{12} - x_6   \equiv 8412 &                  1\\
    x_{13}\equiv 7897 \\
    x_{14}\equiv 9984 & x_{14} - x_7   \equiv 8465 &                  1\\
    x_{15}\equiv 4919 \\
    x_{16}\equiv 2389 & x_{16} - x_8   \equiv 1249 &                  1\\
    x_{17}\equiv 5241 \\
    x_{18}\equiv 4998 & x_{18} - x_9   \equiv 6614 &                  1\\
    x_{19}\equiv 2986 \\
    x_{20}\equiv 8679 & x_{20} - x_{10}\equiv 10235&                  1\\
    x_{21}\equiv 2562 \\
    x_{22}\equiv 4865 & x_{22} - x_{11}\equiv 6319 &                 71
  \end{array}
\]
We have found another prime factor $d = 71$,
and the quotient $\dfrac{201001}{19\times 71} = 149$ is also prime,\\
so $n = 201001 = 19\times 71\times 149$.


\bigskip\noindent{\bf 77.}
Let us now first use Fermat's Method:
For $n = 92131$, $\lceil\sqrt{n}\rceil = $, so
\[
  \begin{array}{cccc}
     t  &\  2t + 1 \ &\  s^2 = t^2 -n\  & s\in\mathbb{Z}\text{?} \\\hline
    304 &    609     &      285         & \times \\
    305 &    611     &      894         & \times \\
    306 &    613     &     1505         & \times \\
    307 &    615     &     2118         & \times \\
    308 &    617     &     2733         & \times \\
    309 &    619     &     3350         & \times \\
    310 &    621     &     3969         & \checkmark
  \end{array}
\]
so  $t = 310$ 
and $s = \sqrt{3969} = 63$, 
and $a = s + t = 373$ 
and $b = t - s = 247$;
hence, $92131 = n = ab = 373\times 247$.
The number 373 is prime but 247 is not; 
we need to factorise the latter:
$\lceil\sqrt{247}\rceil = 16$, so
\[
  \begin{array}{cccc}
     t  &\  2t + 1 \ &\  s^2 = t^2 -n\  & s\in\mathbb{Z}\text{?} \\\hline
     16 &     33     &       9          & \checkmark
  \end{array}
\]
so  $t = 16$ and $s = \sqrt{9} = 3$, 
and $a = s + t = 19$ and $b = t - s = 13$;
these are prime factors,\\
so $92131 = 13\times 19\times 373$.

Let us now use the Pollard $\rho$-Method:
\[
  \begin{array}{llc}
    x_0   \equiv 2     &                        & \gcd(x_{2i} - x_i, n)\\
    x_1   \equiv 5\\
    x_2   \equiv 26    & x_2 - x_1 \equiv    21 &                  1\\
    x_3   \equiv 677\\
    x_4   \equiv 89806 & x_4 - x_2 \equiv 89780 &                  1\\
    x_5   \equiv 62028\\
    x_6   \equiv 82225 & x_6 - x_3 \equiv 81548 &                 19
  \end{array}
\]
We have found a factor $d = 19$ and factor the quotient $\dfrac{n}{d} = 4849$:
\[
  \begin{array}{llc}
    x_0   \equiv    2 &                       & \gcd(x_{2i} - x_i, n)\\
    x_1   \equiv    5 \\
    x_2   \equiv   26 & x_2 - x_1 \equiv   21 &                  1\\
    x_3   \equiv  677 \\
    x_4   \equiv 2524 & x_4 - x_2 \equiv 2498 &                  1\\
    x_5   \equiv 3840 \\
    x_6   \equiv 4641 & x_6 - x_3 \equiv 3964 &                  1\\
    x_7   \equiv 4473 \\
    x_8   \equiv  756 & x_8 - x_4 \equiv 3081 &                 13
  \end{array}
\]
We have found another prime factor $d = 13$,
and $\dfrac{92131}{19\times 13} = 373$ is also prime,
so $n = 92131 = 13\times 19\times 373$.

Finally, we use Shor's Algorithm on $n = 92131$: 
$3^{558}\equiv 1\pmod{n}$,
so
$n | (3^{558}-1) = (3^{279} - 1)(3^{279} + 1)$.\\
Now, it is possible to calculate $\gcd(n,3^{279} - 1) = 13$ 
and $\gcd(n,3^{279} + 1) = 7087$ to find the factors 13 
and $7087 = 19\times 373$.\\
Again, we see that $92131 = 13\times 19\times 373$.


\bigskip\noindent{\bf 78.}
\begin{itemize}
  \item[a)]\moveup
    \[
      \begin{array}{cccc}
         t  &\  2t + 1 \ &\  s^2 = t^2 - N\ & s\in\mathbb{Z}\text{?} \\\hline
        157 &    315     &      152         & \times \\
        158 &    317     &      467         & \times \\
        159 &    319     &      784         & \checkmark
      \end{array}
    \]
    so  $t = 159$
    and $s = \sqrt{784} = 28$,
    and $a = s + t = 187$
    and $b = t - s = 131$.\\
    Therefore, $24497 = N = ab = 131\times 187$.
  \item[b)] Write  $N = 131 = 2^st+1$ with $s = 1$ and $t = 65$.
    Since %$a^t = 2^{65}\equiv -1 \pmod{131}$
    $a^{2^r}t = a^t \equiv -1\pmod{131}$ for $r = 0$,\\
    $n = 131$ is a strong pseudo-prime base~2
    and therefore quite likely to be a prime.
  \item[c)] Note that $2^{40}\equiv 1\pmod{187}$,
    so $187 | (2^{40} - 1) = (2^{20} - 1)(2^{20} + 1)$.\\
    Calculating $\gcd(187,2^{20} - 1) = 11$
    and $\gcd(n,2^{20} + 1) = 17$ to find the factors 11 and 17;\\
    these are prime and are thus the factors of 187.
\end{itemize}


\bigskip\noindent{\bf 79.}
A very simple method could just be to test the $p$ smallest primes $2,3,5,\ldots,p$,
to see whether $\dfrac{n}{2^p-1}$ is an (odd) integer, and if so, whether it is of the form $2^q-1$ for some prime $q$.\\
For $n = 16646017$, we find that $p = 17$: 
\[ 
  \frac{n}{2^{17}-1} = 127 = 2^7-1
\]
Here, $q = 7$ is prime, and so $n = (2^{17}-1)(2^7-1) = 131071 \times 127$.


\newpage\noindent{\bf 80.}
\begin{itemize}
  \item[{\bf a.}] 0,1,0,1,1
  \item[{\bf b.}] As soon as any consecutive 3 bits $x_i,x_{i+1},x_{i+1}$ reappear later in the list,
    the output will repeat itself.
    The numbers $x_0,\ldots,x_9$ are as follows:
    \[
      x_0 = 1, 1, 0, 0, 1, 0, 1, (x_7 =)\, 1, 1, 0
    \]
    We see that the first sequence 1,1,0 reappears 7 steps later;
    7 is therefore the period of this LSFR.
\end{itemize}


\bigskip\noindent{\bf 81.}
\begin{itemize}
  \item[{\bf a.}]
    The output numbers $x_0,\ldots,x_{19}$ are as follows:
    \[
      \begin{array}{|c|cccccccccccccccccccc|}\hline
          i & 0 & 1 & 2 & 3 &  4 & 5 & 6 &  7 & 8 &  9 & 10 & 11 & 12 & 13 & 14 & 15 & 16 & 17 & 18 & 19\\\hline
        x_i & 1 & 8 & 3 & 4 & 11 & 6 & 7 & 14 & 9 & 10 & 17 & 12 & 13 &  2 & 15 & 16 &  5 &  0 &  1 &  8\\\hline
      \end{array}
    \]
    As soon as any number reappears later in the list,
    the output will repeat itself.
    We see that $x_0 = 1$ reappears as $x_{18} = 1$; 
    the period of this LSFR is therefore 18.
  \item[{\bf b.}] 
    The output numbers $x_0,\ldots,x_{19}$ are as follows:
    \[
      \begin{array}{|c|cccccccccccccccccccc|}\hline
          i & 0 & 1 & 2 & 3 & 4 & 5 & 6 & 7 & 8 & 9 & 10 & 11 & 12 & 13 & 14 & 15 & 16 & 17 & 18 & 19\\\hline
        x_i & 1 & 0 & 0 & 0 & 1 & 1 & 1 & 1 & 0 & 1 &  0 &  1 &  1 &  0 &  0 &  1 &  0 &  0 &  0 &  1\\\hline
      \end{array}
    \]
    As soon as any consecutive 3 bits $x_i,\ldots,x_{i+4}$ reappear later in the list,
    the output will repeat itself.
    We see that the first sequence 1,0,0,0 reappears 15 steps later;
    15 is therefore the period of this LSFR.
\end{itemize}


\bigskip\noindent{\bf 82.}
\begin{itemize}
  \item[{\bf a.}]
    First, write the message as it has been received:
    \begin{verbatim}
      FSOTU
      OHFOI
      UIJNP
      RPUTM
      TELHE
      HQYEN
    \end{verbatim}
    You can use a number of easy heuristic arguments 
    (for instance, 
    ``fourth" goes with ``of July", as do ``q" and ``u") 
    to permute the columns above in the right way 
    but will get the following correct answer:
    \begin{verbatim}
      SUTFO
      HIOOF
      IPNUJ
      PMTRU
      EEHTL
      QNEHY
    \end{verbatim}
    This gives us the message ``SHIP EQUIPMENT ON THE FOURTH OF JULY".
  \item[{\bf b.}]
    $\sigma 
    = \vc{
        1 & 2 & 3 & 4 & 5\\
        4 & 1 & 5 & 3 & 2}$
  \item[{\bf c.}]
    First, write the message as it has been received:
    \begin{verbatim}
      SLAFK
      EOROE
      LIERO
      LLSEV
      ASBTE
      LHEAR
    \end{verbatim}
    Now permute the columns:
    \begin{verbatim}
      LKFSA
      OEOER
      IORLE
      LVELS
      SETAB
      HRALE
    \end{verbatim}
    The enciphered message is then ``LKFSA OEOER IORLE LVELS SETAB HRALE".
\end{itemize}
%FSKAL OEERO RLOEU ELVSL TAEBS ALREH

\bigskip\noindent{\bf 83.}
Looking at the first block HSTII, we might guess that 
it was enciphered from ``THIS I",
using one of the permutations
%THISI
%HSTII
\[
  \sigma_1
    = \vc{
        1 & 2 & 3 & 4 & 5\\
        3 & 1 & 4 & 2 & 5}
  \qquad\text{and}\qquad
  \sigma_2
    = \vc{
        1 & 2 & 3 & 4 & 5\\
        3 & 1 & 5 & 2 & 4}
\]
Applying the inverse $\sigma_1^{-1}$ of 
the first permutation to the 1st three blocks, 
we get ``THIS IS A TON HERAR".
This does not seem to make sense,
so lets try the inverse $\sigma_2^{-1} = \vc{
        1 & 2 & 3 & 4 & 5\\
        2 & 4 & 1 & 5 & 3}
$ of the second permutation:\\
We now get a much more lucid message:
\verb|THIS IS ANOTHER ARTICLE ON SECRET CODES PQ|

\end{document}

